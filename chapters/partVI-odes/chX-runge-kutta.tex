
\chapter{Runge-Kutta Methods}
\label{chap:rk}

\section{Introduction and Motivation}
\label{sec:rk:intro}
Runge-Kutta (RK) methods are a family of one-step methods for solving initial value problems of the form
\begin{equation}
  y'(t) = f(t, y(t)), \qquad y(t_0) = y_0.
\end{equation}
Unlike linear multistep methods, RK methods use only the current value $(t_n, y_n)$ to advance the solution, but achieve high order by evaluating $f$ at several intermediate points within each step. The classical fourth-order Runge-Kutta method (RK4) is especially popular due to its balance of accuracy and simplicity.

\section{General Formulation of s-Stage Runge-Kutta Methods}
\label{sec:rk:general}
An $s$-stage Runge-Kutta method advances the solution from $y_n$ to $y_{n+1}$ as follows:
\begin{align*}
  k_i &= f\left(t_n + c_i h,\; y_n + h \sum_{j=1}^s a_{ij} k_j\right), \qquad i = 1, \ldots, s, \\
  y_{n+1} &= y_n + h \sum_{i=1}^s b_i k_i.
\end{align*}
The coefficients $a_{ij}$, $b_i$, and $c_i$ fully determine the method.

\section{The Butcher Tableau}
\label{sec:rk:butcher}
The coefficients of a Runge-Kutta method are conveniently summarized in the \emph{Butcher tableau}:
\begin{equation*}
\begin{array}{c|ccc}
  c_1 & a_{11} & \cdots & a_{1s} \\
  \vdots & \vdots & & \vdots \\
  c_s & a_{s1} & \cdots & a_{ss} \\
  \hline
      & b_1    & \cdots & b_s
\end{array}
\end{equation*}
where $a_{ij}$, $b_i$, and $c_i$ are the method's parameters. Explicit methods have $a_{ij} = 0$ for $j \geq i$.

\subsection{Example: Classical RK4}
The classical fourth-order Runge-Kutta method (RK4) has the tableau:
\begin{equation*}
\begin{array}{c|cccc}
  0   & 0   & 0   & 0   & 0   \\
  1/2 & 1/2 & 0   & 0   & 0   \\
  1/2 & 0   & 1/2 & 0   & 0   \\
  1   & 0   & 0   & 1   & 0   \\
  \hline
      & 1/6 & 1/3 & 1/3 & 1/6
\end{array}
\end{equation*}
This method is widely used for its accuracy and simplicity.
\section{Order Conditions and Properties}
\label{sec:rk:order}
The accuracy of a Runge-Kutta method is determined by its coefficients. A method is said to be of order $p$ if its local truncation error is $\mathcal{O}(h^{p+1})$. Achieving higher order requires more stages and the satisfaction of increasingly intricate algebraic conditions, known as Butcher's order conditions.

For a Runge-Kutta method to be \emph{consistent} (i.e., at least first-order accurate), the coefficients must satisfy
\begin{equation}
  \sum_{i=1}^s b_i = 1,
\end{equation}
which ensures the method reproduces the exact solution for constant functions.

More generally, the order conditions up to $p=4$ are as follows:
\begin{itemize}
  \item \textbf{Order 1 (Consistency):}
    \begin{equation}
      \sum_{i=1}^s b_i = 1
    \end{equation}
  \item \textbf{Order 2:}
    \begin{equation}
      \sum_{i=1}^s b_i c_i = \frac{1}{2}
    \end{equation}
  \item \textbf{Order 3:}
    \begin{align}
      \sum_{i=1}^s b_i c_i^2 &= \frac{1}{3} \\
      \sum_{i=1}^s b_i \sum_{j=1}^s a_{ij} c_j &= \frac{1}{6}
    \end{align}
  \item \textbf{Order 4:}
    \begin{align}
      \sum_{i=1}^s b_i c_i^3 &= \frac{1}{4} \\
      \sum_{i=1}^s b_i c_i \sum_{j=1}^s a_{ij} c_j &= \frac{1}{8} \\
      \sum_{i=1}^s b_i \left(\sum_{j=1}^s a_{ij} c_j^2\right) &= \frac{1}{12} \\
      \sum_{i=1}^s b_i \sum_{j=1}^s a_{ij} \sum_{k=1}^s a_{jk} c_k &= \frac{1}{24}
    \end{align}
\end{itemize}

\section{Stability and Applications}
\label{sec:rk:stability}
Stability analysis for RK methods often involves applying the method to the test equation $y' = \lambda y$. The stability region is the set of $z = \lambda h$ for which the numerical solution does not grow. Explicit RK methods are not suitable for stiff problems, while implicit RK methods (e.g., Gauss, Radau, Lobatto) are designed for such cases.

\section{Exercises and Solutions}

\subsection*{Problem 5. (Ordinary Differential Equations)}

\textbf{a) Stages and Formulation from Butcher Tableau}

Given the tableau (reformatted for clarity):
\[
\begin{array}{c|cccc}
0   & 0   & 0   & 0   & 0   \\
\frac{1}{2} & \frac{1}{2} & 0   & 0   & 0   \\
\frac{1}{2} & 0   & \frac{1}{2} & 0   & 0   \\
1   & 0   & 0   & 1   & 0   \\
\hline
  & \frac{1}{6} & \frac{1}{3} & \frac{1}{3} & \frac{1}{6}
\end{array}
\]
There are $s=4$ stages.

The method is:
\begin{align*}
K_1 &= f(t_n, y_n) \\
K_2 &= f\left(t_n + \frac{1}{2}h,\, y_n + \frac{1}{2}h K_1\right) \\
K_3 &= f\left(t_n + \frac{1}{2}h,\, y_n + \frac{1}{2}h K_2\right) \\
K_4 &= f\left(t_n + h,\, y_n + h K_3\right) \\
y_{n+1} &= y_n + \frac{h}{6}(K_1 + 2K_2 + 2K_3 + K_4)
\end{align*}

\textbf{b) Order 3 Conditions for 3-stage RK}

Given tableau:
\[
\begin{array}{c|ccc}
0 & 0 & 0 & 0 \\
\frac{1}{2} & \frac{1}{2} & 0 & 0 \\
1 & a_{31} & a_{32} & 0 \\
\hline
  & b_1 & b_2 & b_3
\end{array}
\]
Order 3 conditions:
\begin{align*}
b_1 + b_2 + b_3 &= 1 \\
b_2 \cdot \frac{1}{2} + b_3 \cdot 1 &= \frac{1}{2} \\
b_2 \cdot \left(\frac{1}{2}\right)^2 + b_3 \cdot 1^2 &= \frac{1}{3} \\
b_3 a_{32} \cdot \frac{1}{2} &= \frac{1}{6}
\end{align*}
Gives the coefficients:
\[
a_{31} = -1,\quad a_{32} = 2,\quad b_1 = \frac{1}{6},\quad b_2 = \frac{2}{3},\quad b_3 = \frac{1}{6}
\]

\textbf{c) Maximum Order for Explicit RK with $s=4$}

For the test equation $y' = y$, $y(0) = y_0$, the Taylor expansion of $y(h)$ is:
\[
y(h) = y_0 + h y_0 + \frac{h^2}{2} y_0 + \frac{h^3}{6} y_0 + \frac{h^4}{24} y_0 + \cdots
\]
An explicit $s$-stage RK method can match the Taylor expansion up to $h^s$ but not higher, since it only uses $s$ function evaluations per step. Thus, the order cannot exceed $s$.

\section{Summary}
Runge-Kutta methods provide a flexible and powerful class of one-step methods for ODEs. The Butcher tableau offers a compact notation for their coefficients. Explicit RK methods are easy to implement and widely used for non-stiff problems, while implicit variants are important for stiff systems.
