\chapter{Fourier Analysis}
\label{ch:fourier}

\section{Fourier Series}
\label{sec:fourier-series}

A Fourier series represents any reasonably well--behaved \(2L\)--periodic function \(f(x)\) as an infinite linear combination of sines and cosines.

\subsection{Orthogonal Basis of Sines and Cosines}
The key idea is that the set
\begin{equation}
    \mathcal{S} := \left\{1,\cos\tfrac{\pi x}{L},\sin\tfrac{\pi x}{L},\cos\tfrac{2\pi x}{L},\sin\tfrac{2\pi x}{L},\dots \right\}
\end{equation}
forms an orthogonal basis on the interval \([-L,L]\).

Because sines and cosines of different frequencies are orthogonal, cross--terms disappear when computing the coefficients:

\begin{proposition}{Orthogonality of Sines and Cosines}{orthogonality}
    For integers \( m, n \geq 0 \) on the interval \([-L, L]\), the following orthogonality relations hold:
    \begin{align*}
        \int_{-L}^{L} \cos\left(\tfrac{m\pi x}{L}\right) \cos\left(\tfrac{n\pi x}{L}\right) \, dx
         & =
        \begin{cases}
            0,  & m \neq n,     \\
            L,  & m = n \neq 0, \\
            2L, & m = n = 0,
        \end{cases}                 \\
        \int_{-L}^{L} \sin\left(\tfrac{m\pi x}{L}\right) \sin\left(\tfrac{n\pi x}{L}\right) \, dx
         & =
        \begin{cases}
            0, & m \neq n,     \\
            L, & m = n \neq 0,
        \end{cases}                  \\
        \int_{-L}^{L} \cos\left(\tfrac{m\pi x}{L}\right) \sin\left(\tfrac{n\pi x}{L}\right) \, dx
         & = 0 \qquad \text{for all } m, n.
    \end{align*}
\end{proposition}

These relations show that the set of functions \( \{1, \cos(\frac{n\pi x}{L}), \sin(\frac{n\pi x}{L}) \} \) forms an orthogonal basis for \( L^2([-L, L]) \).

\subsection{Fourier Series Definition}
If we project \(f\) onto these basis functions, we obtain its Fourier expansion. The coefficients are computed by taking the inner product of \(f\) with each basis function, exploiting their orthogonality.

\begin{definition}{Fourier Series}{fourier-series}
    Let \(f\) be a \(2L\)--periodic function integrable on \([-L,L]\).  Its Fourier series is
    \begin{equation}
        f(x) = \frac{a_0}{2} +\sum_{n=1}^{\infty} \left(a_n\cos\tfrac{n\pi x}{L} +b_n\sin \tfrac{n\pi x}{L}\right)
    \end{equation}

    where the coefficients are the orthogonal projections
    \begin{align*}
        a_0 & =\frac{1}{L}\int_{-L}^{L}f(x)\, dx                      \\
        a_n & =\frac{1}{L}\int_{-L}^{L}f(x)\cos\tfrac{n\pi x}{L}\, dx \\
        b_n & =\frac{1}{L}\int_{-L}^{L}f(x)\sin\tfrac{n\pi x}{L}\, dx
    \end{align*}
\end{definition}

Intuitively, each coefficient \(a_n\) (resp.\ \(b_n\)) measures how much of the cosine (resp.\ sine) wave of frequency \(n\pi/L\) is present in \(f\).

The Fourier series can be understood geometrically as projecting the function \(f\) onto the orthogonal basis of trigonometric functions. Each coefficient represents the "component" of \(f\) in the direction of a particular basis function.

\begin{figure}[htbp]
    \centering
    \includestandalone[width=0.48\textwidth]{figures/fourier-ortho-1}
    \hfill
    \includestandalone[width=0.44\textwidth]{figures/fourier-ortho-2}
    \caption{
        Left: Decomposition of a function \(f(x)\) into its trigonometric components.
        Right: The orthogonal trigonometric basis functions on \([-L, L]\).
    }
    \label{fig:fourier-orthogonal-projection}
\end{figure}

The projection formula for coefficient \(a_n\) can be visualized as:
\begin{equation}
    a_n = \frac{\langle f, \cos(n\pi x/L) \rangle}{\|\cos(n\pi x/L)\|^2} = \frac{\int_{-L}^{L} f(x)\cos(n\pi x/L)\,dx}{\int_{-L}^{L} \cos^2(n\pi x/L)\,dx} = \frac{1}{L}\int_{-L}^{L} f(x)\cos(n\pi x/L)\,dx
\end{equation}

This geometric interpretation shows that Fourier analysis is fundamentally about finding the best approximation to \(f\) in the subspace spanned by trigonometric functions, where "best" means minimizing the \(L^2\) error over the interval \([-L,L]\).

\begin{definition}{Complex Fourier Series}{complex-fourier}
    The complex exponential form of the Fourier series is:
    \begin{equation}
        f(x) = \sum_{n=-\infty}^{\infty} c_n e^{i n \pi x / L}
    \end{equation}
    where the complex coefficients are:
    \begin{equation}
        c_n = \frac{1}{2L} \int_{-L}^{L} f(x) e^{-i n \pi x / L} \, dx
    \end{equation}

    The relationship between real and complex coefficients is:
    \begin{align}
        c_0    & = \frac{a_0}{2}                                        \\
        c_n    & = \frac{a_n - i b_n}{2} \quad (n > 0)                  \\
        c_{-n} & = \frac{a_n + i b_n}{2} = \overline{c_n} \quad (n > 0)
    \end{align}
\end{definition}

\subsection{Properties of Fourier Series}
The Fourier series possesses several important properties that mirror the structure of the underlying function space.
\begin{property}{Properties of Fourier Series}{fourier-properties}
    Let \( f(x) \) and \( g(x) \) be \( 2L \)-periodic functions with Fourier series, and let \( \alpha, \beta \) be constants. Then:
    \begin{enumerate}[label=\textbf{\arabic*.}, leftmargin=*, noitemsep]
        \item \textbf{Linearity:} The Fourier series is linear:
              \[
                  \mathcal{F}[\alpha f(x) + \beta g(x)] = \alpha \mathcal{F}[f(x)] + \beta \mathcal{F}[g(x)].
              \]
        \item \textbf{Differentiation:} If \( f \) is continuous and piecewise smooth, the Fourier series of \( f'(x) \) is obtained by term-wise differentiation:
              \[
                  f'(x) = \sum_{n=1}^{\infty} \left( -a_n \frac{n\pi}{L} \sin\left(\frac{n\pi x}{L}\right) + b_n \frac{n\pi}{L} \cos\left(\frac{n\pi x}{L}\right) \right),
              \]
              where \( a_n, b_n \) are the Fourier coefficients of \( f \).
        \item \textbf{Integration:} The Fourier series of \( f(x) \) can be integrated term by term:
              \[
                  \int_{-L}^{x} f(t)\,dt = \frac{a_0}{2}(x + L) + \sum_{n=1}^{\infty} \left[ a_n \frac{L}{n\pi} \sin\left(\frac{n\pi x}{L}\right) - b_n \frac{L}{n\pi} \cos\left(\frac{n\pi x}{L}\right) + b_n \frac{L}{n\pi} \right],
              \]
              where the integration constant is chosen so that the result is \( 2L \)-periodic.
    \end{enumerate}
\end{property}

\subsubsection{Parseval's Theorem}
Parseval's theorem relates the \(L^2\) norm of a function to the sum of the squares of its Fourier coefficients, providing a powerful tool for analyzing the energy content of signals.
\begin{theorem}{Parseval's Theorem}{parseval}
    Let \( f \) be a \( 2L \)-periodic, square-integrable function on \([-L, L]\), with complex Fourier coefficients
    \[
        c_n = \frac{1}{2L} \int_{-L}^{L} f(x) e^{-i n \pi x / L} \, dx.
    \]
    Define the inner product and norm on \( \mathrm{L}^2([-L, L]) \) by
    \[
        \langle f, g \rangle = \int_{-L}^{L} f(x) \cdot \overline{g(x)} \, dx, \qquad \|f\|^2 = \langle f, f \rangle = \int_{-L}^{L} |f(x)|^2 \, dx.
    \]
    Then Parseval's theorem states:
    \[
        \|f\|^2 = \int_{-L}^{L} |f(x)|^2 \, dx = 2L \sum_{n=-\infty}^{\infty} |c_n|^2.
    \]
    This expresses the completeness and orthogonality of the exponential basis in \( L^2([-L, L]) \).
\end{theorem}

\begin{proof}
    Let \( f(x) = \sum_{n=-\infty}^{\infty} c_n e^{i n \pi x / L} \) be the complex Fourier series of \( f \), converging in \( L^2([-L, L]) \).
    The coefficients are given by
    \[
        c_n = \frac{1}{2L} \int_{-L}^{L} f(x) e^{-i n \pi x / L} \, dx.
    \]
    The exponential functions \( \{e^{i n \pi x / L}\}_{n \in \mathbb{Z}} \) form an orthogonal basis for \( L^2([-L, L]) \) with
    \[
        \int_{-L}^{L} e^{i n \pi x / L} \overline{e^{i m \pi x / L}} \, dx = \int_{-L}^{L} e^{i (n-m) \pi x / L} \, dx =
        \begin{cases}
            2L, & n = m,    \\
            0,  & n \neq m.
        \end{cases}
    \]
    Therefore,
    \begin{align*}
        \|f\|^2 & = \int_{-L}^{L} |f(x)|^2 \, dx                                                                                                                                 \\
                & = \int_{-L}^{L} \left( \sum_{n=-\infty}^{\infty} c_n e^{i n \pi x / L} \right) \overline{ \left(\sum_{m=-\infty}^{\infty} c_m e^{i m \pi x / L} \right)} \, dx \\
                & = \int_{-L}^{L} \sum_{n} \sum_{m} c_n \overline{c_m} e^{i (n-m) \pi x / L} \, dx                                                                               \\
                & = \sum_{n} \sum_{m} c_n \overline{c_m} \int_{-L}^{L} e^{i (n-m) \pi x / L} \, dx                                                                               \\
                & = \sum_{n} \sum_{m} c_n \overline{c_m} \cdot 2L \, \delta_{nm}                                                                                                 \\
                & = 2L \sum_{n=-\infty}^{\infty} |c_n|^2.
    \end{align*}
    This completes the proof.
\end{proof}

\subsubsection{Gibbs Phenomenon}
When approximating a function with discontinuities using Fourier series, the partial sums exhibit an overshoot near the discontinuities that does not disappear as more terms are added. This overshoot is approximately 9\% of the jump discontinuity and is known as the \textbf{Gibbs phenomenon}.

\begin{figure}[H]
    \centering
    \includestandalone[width=0.8\textwidth]{figures/fourier-gibbs}
    \caption{Gibbs phenomenon showing persistent overshoot near discontinuities}
\end{figure}

\subsubsection{Convergence of Fourier Series}
A natural question is: \emph{When does the Fourier series of a function converge to the function itself?}
Not every periodic function has a Fourier series that converges everywhere to the original function.
The answer is provided by the \textbf{Dirichlet conditions}, which give sufficient (but not necessary) criteria for pointwise convergence of the Fourier series.

\begin{theorem}{Dirichlet Conditions for Fourier Series Convergence}{dirichlet-conditions}
    Let \( f \) be a \( 2L \)-periodic, piecewise smooth function on \(\mathbb{R}\). If the following conditions hold:
    \begin{enumerate}
        \item \( f \) is absolutely integrable over one period: \( \int_{-L}^{L} |f(x)|\,dx < \infty \).
        \item \( f \) has only finitely many discontinuities (all of finite size) and finitely many extrema in any period.
    \end{enumerate}
    Then, at every point \( x \) where \( f \) is continuous, its Fourier series converges to \( f(x) \). At a point of discontinuity \( x_0 \), the Fourier series converges to the average of the right and left limits:
    \begin{equation}
        \lim_{N \to \infty} S_N(x_0) = \frac{f(x_0^+) + f(x_0^-)}{2}
    \end{equation}
    where \( S_N(x) \) is the \( N \)-th partial sum of the Fourier series, and \( f(x_0^+), f(x_0^-) \) are the right and left limits of \( f \) at \( x_0 \).
\end{theorem}

\begin{itemize}
    \item Dirichlet conditions are sufficient (not necessary) for pointwise convergence.
    \item At discontinuities, the series converges to the average of left and right limits.
    \item Convergence is pointwise, not uniform; Gibbs phenomenon may appear near jumps.
\end{itemize}

\subsection{Examples of Fourier Series}
\begin{example}{Square Wave}{square-wave}
    Consider the square wave function defined on
    \begin{equation}
        f(x) = \begin{cases}
            1  & \text{if } 0 < x < \pi   \\
            -1 & \text{if } -\pi < x < 0  \\
            0  & \text{if } x = 0, \pm\pi
        \end{cases}
    \end{equation}

    Since the function is odd, all cosine terms vanish (\(a_n = 0\)), and we only have sine terms:
    \begin{align}
        b_n & = \frac{1}{\pi} \int_{-\pi}^{\pi} f(x) \sin(nx) \, dx    \\
            & = \frac{2}{\pi} \int_{0}^{\pi} \sin(nx) \, dx            \\
            & = \frac{2}{\pi} \left[-\frac{\cos(nx)}{n}\right]_0^{\pi} \\
            & = \frac{2}{n\pi}(1 - \cos(n\pi))                         \\
            & = \begin{cases}
                    \frac{4}{n\pi} & \text{if } n \text{ is odd}  \\
                    0              & \text{if } n \text{ is even}
                \end{cases}
    \end{align}

    Therefore, the Fourier series is:
    \begin{equation}
        f(x) = \frac{4}{\pi} \sum_{n=1,3,5,\ldots} \frac{\sin(nx)}{n} = \frac{4}{\pi}\left(\sin x + \frac{\sin 3x}{3} + \frac{\sin 5x}{5} + \cdots\right)
    \end{equation}
\end{example}

\begin{figure}[H]
    \centering
    \includestandalone[width=0.8\textwidth]{figures/fourier-square}
    \caption{Fourier series approximation of a square wave showing convergence with increasing number of terms}
\end{figure}

\begin{example}{Sawtooth Wave}{sawtooth-wave}
    Consider the sawtooth wave defined on \((-\pi, \pi)\):
    \begin{equation}
        f(x) = x \quad \text{for } x \in (-\pi, \pi)
    \end{equation}

    Since this function is odd, we have \(a_n = 0\) for all \(n\), and:
    \begin{align}
        b_n & = \frac{1}{\pi} \int_{-\pi}^{\pi} x \sin(nx) \, dx \\
            & = \frac{2}{\pi} \int_{0}^{\pi} x \sin(nx) \, dx    \\
            & = \frac{2(-1)^{n+1}}{n}
    \end{align}

    The Fourier series is:
    \begin{equation}
        f(x) = 2 \sum_{n=1}^{\infty} \frac{(-1)^{n+1}}{n} \sin(nx) = 2\left(\sin x - \frac{\sin 2x}{2} + \frac{\sin 3x}{3} - \cdots\right)
    \end{equation}
\end{example}

\begin{figure}[H]
    \centering
    \includestandalone[width=0.8\textwidth]{figures/fourier-sawtooth}
    \caption{Fourier series approximation of a sawtooth wave}
    \label{fig:fourier-sawtooth}
\end{figure}
