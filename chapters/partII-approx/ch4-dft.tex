\section{Discrete Fourier Transform (DFT)}

\subsection{Motivation: From Fourier Series to DFT}
The Discrete Fourier Transform (DFT) arises naturally when we sample a periodic function at finitely many points. While the classical Fourier series decomposes a function into sines and cosines (or complex exponentials) over a continuous interval, in practice we often work with discrete data: sampled signals, digital images, or numerical solutions. The DFT provides a way to analyze the frequency content of such discrete data, mirroring the role of the Fourier series for continuous functions.

\subsection{Definition and Properties of the DFT}
\begin{definition}{Discrete Fourier Transform}{dft}
    Given a sequence $\{f_k\}_{k=0}^{N-1}$ (e.g., $N$ samples of a periodic function), the DFT produces a sequence $\{F_n\}_{n=0}^{N-1}$:
    \begin{equation}
        F_n = \sum_{k=0}^{N-1} f_k \, e^{-2\pi i n k / N}, \qquad n = 0, 1, \ldots, N-1.
    \end{equation}
    The inverse DFT reconstructs the original sequence:
    \begin{equation}
        f_k = \frac{1}{N} \sum_{n=0}^{N-1} F_n \, e^{2\pi i n k / N}, \qquad k = 0, 1, \ldots, N-1.
    \end{equation}
\end{definition}

The DFT can be written as a matrix-vector multiplication:
\begin{equation}
    \mathbf{F} = \mathcal{F}_N \mathbf{f}
\end{equation}
where $\mathcal{F}_N$ is the $N \times N$ DFT matrix with entries $e^{-2\pi i jk / N}$.

\begin{definition}{DFT Matrix}{dft-matrix}
    The DFT matrix $\mathcal{F}_N$ is defined by
    \begin{equation}
        (\mathcal{F}_N)_{jk} = e^{-2\pi i jk / N} = \omega_N^{jk}, \qquad \omega_N = e^{-2\pi i / N}, \quad j, k = 0, \ldots, N-1.
    \end{equation}
    This matrix is a Vandermonde matrix built from the $N$th roots of unity.
\end{definition}

\paragraph{Key Properties}
\begin{itemize}
    \item \textbf{Linearity:} The DFT is a linear transformation.
    \item \textbf{Periodicity:} Both input and output sequences are periodic with period $N$.
    \item \textbf{Orthogonality:} The complex exponentials $e^{2\pi i n k / N}$ form an orthogonal basis for $\mathbb{C}^N$.
    \item \textbf{Parseval's Theorem (Discrete):}
          \[
              \sum_{k=0}^{N-1} |f_k|^2 = \frac{1}{N} \sum_{n=0}^{N-1} |F_n|^2
          \]
\end{itemize}

\subsection{Fast Fourier Transform (FFT)}

For large $N$, direct computation of the DFT via matrix multiplication requires $O(N^2)$ operations. The \textbf{Fast Fourier Transform (FFT)} is a family of algorithms that compute the DFT much more efficiently, in $O(N \log N)$ time, by exploiting symmetries in the roots of unity.

\subsubsection{Recursive Divide-and-Conquer Structure}
The most common FFT algorithm (Cooley--Tukey) recursively splits the DFT of a sequence of length $N$ (assumed even) into two DFTs of length $N/2$: one for the even-indexed elements and one for the odd-indexed elements. These are then combined using so-called \emph{butterfly operations}.

\begin{algorithm}[H]
    \caption{Radix-2 Cooley--Tukey FFT Algorithm}
    \begin{algorithmic}[1]
        \Require Sequence $f = (f_0, \ldots, f_{N-1})$ of length $N = 2^m$
        \Ensure DFT $F = (F_0, \ldots, F_{N-1})$
        \Function{FFT}{$f$}
        \If{$N = 1$}
        \State \Return $f$
        \Else
        \State Split $f$ into even and odd indexed parts:
        \State $f^{(\mathrm{even})}_k = f_{2k}$
        \State $f^{(\mathrm{odd})}_k = f_{2k+1}$
        \State $E \gets$ \Call{FFT}{$f^{(\text{even})}$}
        \State $O \gets$ \Call{FFT}{$f^{(\text{odd})}$}
        \For{$n = 0$ \textbf{to} $N/2-1$}
        \State $\omega \gets e^{-2\pi i n / N}$ \hfill (\textbf{twiddle factor})
        \State $F_n \gets E_n + \omega \cdot O_n$
        \State $F_{n+N/2} \gets E_n - \omega \cdot O_n$
        \EndFor
        \State \Return $F$
        \EndIf
        \EndFunction
    \end{algorithmic}
\end{algorithm}

\paragraph{Twiddle Factor} The \textbf{twiddle factor} $e^{-2\pi i n / N}$ is the complex exponential that introduces the necessary phase shift when combining the even and odd DFTs.

\paragraph{Butterfly Operation} The basic computation in the FFT is the \emph{butterfly operation}, which combines two values (from the even and odd DFTs) using the twiddle factor:
\[
    \begin{pmatrix}
        F_n \\
        F_{n+N/2}
    \end{pmatrix}
    =
    \begin{pmatrix}
        1 & \mathcal{F}_N^n  \\
        1 & -\mathcal{F}_N^n
    \end{pmatrix}
    \begin{pmatrix}
        E_n \\
        O_n
    \end{pmatrix}
\]
for $n = 0, \ldots, N/2-1$.

\paragraph{Complexity and Efficiency}
The FFT reduces the computational complexity of the DFT from $O(N^2)$ to $O(N \log N)$ by recursively breaking the problem into smaller subproblems and combining their results efficiently.

\subsection{Examples and Interpretation}

\begin{example}{DFT Matrix for $N=4$}{dft-matrix-4}
    For $N = 4$, the primitive 4th root of unity is $\omega_4 = e^{-2\pi i / 4} = i$. The DFT matrix is
    \[
        \mathcal{F}_4 =
        \begin{pmatrix}
            1 & 1  & 1  & 1  \\
            1 & i  & -1 & -i \\
            1 & -1 & 1  & -1 \\
            1 & -i & -1 & i
        \end{pmatrix}
    \]
    Thus, for $\mathbf{f} = (f_0, f_1, f_2, f_3)^\top$, the DFT is $\mathbf{F} = \mathcal{F}_4 \mathbf{f}$.
\end{example}

\begin{example}{DFT of a Simple Sequence}{dft-example}
    Let $N = 4$ and $f_0 = 1, f_1 = 0, f_2 = -1, f_3 = 0$. Compute the DFT:
    \begin{align*}
        F_0 & = 1 + 0 + (-1) + 0 = 0                                                                           \\
        F_1 & = 1 + 0 \cdot e^{-i\pi/2} + (-1) \cdot e^{-i\pi} + 0 \cdot e^{-i3\pi/2} = 1 + 0 + 1 + 0 = 2      \\
        F_2 & = 1 + 0 \cdot e^{-i\pi} + (-1) \cdot e^{-i2\pi} + 0 \cdot e^{-i3\pi} = 1 + 0 - 1 + 0 = 0         \\
        F_3 & = 1 + 0 \cdot e^{-i3\pi/2} + (-1) \cdot e^{-i9\pi/2} + 0 \cdot e^{-i27\pi/2} = 1 + 0 + 1 + 0 = 2
    \end{align*}
    Thus, the DFT is $ (0, 2, 0, 2) $.
\end{example}

\begin{example}{DFT Example: $s = (1, 3, 1, -1)$}{dft-example2}
    Let $N = 4$ and $s = (1, 3, 1, -1) \in \mathbb{R}^4$. The DFT is
    \[
        \hat{s}_k = \sum_{n=0}^{3} s_n e^{-2\pi i k n / 4}, \qquad k = 0,1,2,3.
    \]
    Compute each component:
    \begin{align*}
        \hat{s}_0 & = 1 + 3 + 1 + (-1) = 4                                  \\
        \hat{s}_1 & = 1 + 3 e^{-i\pi/2} + 1 e^{-i\pi} + (-1) e^{-i3\pi/2}   \\
                  & = 1 + 3i - 1 + i = 4i                                   \\
        \hat{s}_2 & = 1 + 3 e^{-i\pi} + 1 e^{-i2\pi} + (-1) e^{-i3\pi}      \\
                  & = 1 - 3 + 1 + 1 = 0                                     \\
        \hat{s}_3 & = 1 + 3 e^{-i3\pi/2} + 1 e^{-i3\pi} + (-1) e^{-i9\pi/2} \\
                  & = 1 - 3i - 1 - i = -4i
    \end{align*}
    So, the DFT is $\hat{s} = (4, 4i, 0, -4i)$.
\end{example}

\paragraph{Interpretation: Relation to Continuous Fourier Coefficients}
The DFT coefficients $\hat{s}_k$ are discrete analogues of the continuous Fourier complex coefficients $c_k(f)$. For a sampled $2L$-periodic function $f(x)$ at $N$ equispaced points, the DFT coefficients approximate $c_k(f)$ up to normalization and scaling. As $N$ increases and the sampling becomes finer, the DFT coefficients converge (after normalization) to the continuous Fourier coefficients.

\subsection{Applications}
DFT and FFT are fundamental tools in:
\begin{itemize}
    \item Signal and image processing (filtering, compression)
    \item Numerical solutions of differential equations (spectral methods)
    \item Data analysis and pattern recognition
\end{itemize}

\subsection{Further Examples and Discussion}

\subsubsection*{DFT Example: $s = (1, 3, 1, -1)$}

Let $N = 4$ and $s = (1, 3, 1, -1) \in \mathbb{R}^4$. The DFT is
\[
    \hat{s}_k = \sum_{n=0}^{3} s_n e^{-2\pi i k n / 4}, \qquad k = 0,1,2,3.
\]
Compute each component:
\begin{align*}
    \hat{s}_0 & = 1 + 3 + 1 + (-1) = 4                                  \\
    \hat{s}_1 & = 1 + 3 e^{-i\pi/2} + 1 e^{-i\pi} + (-1) e^{-i3\pi/2}   \\
              & = 1 + 3i - 1 + i = 4i                                   \\
    \hat{s}_2 & = 1 + 3 e^{-i\pi} + 1 e^{-i2\pi} + (-1) e^{-i3\pi}      \\
              & = 1 - 3 + 1 + 1 = 0                                     \\
    \hat{s}_3 & = 1 + 3 e^{-i3\pi/2} + 1 e^{-i3\pi} + (-1) e^{-i9\pi/2} \\
              & = 1 - 3i - 1 - i = -4i
\end{align*}
So, the DFT is $\hat{s} = (4, 4i, 0, -4i)$.

\paragraph{Relation to Fourier Complex Coefficients}
The DFT coefficients $\hat{s}_k$ are discrete analogues of the continuous Fourier complex coefficients $c_k(f)$. For a sampled $2L$-periodic function $f(x)$ at $N$ equispaced points, the DFT coefficients approximate $c_k(f)$ up to normalization and scaling. Specifically, as $N$ increases and the sampling becomes finer, the DFT coefficients converge (after normalization) to the continuous Fourier coefficients.

\subsubsection*{DFT as Matrix Multiplication and FFT Complexity}
The DFT can be written as a matrix-vector multiplication:
\[
    \hat{s} = F_N s
\]
where $F_N$ is the $N \times N$ matrix with entries $(F_N)_{jk} = e^{-2\pi i jk / N}$.
A general matrix-vector multiplication requires $O(N^2)$ operations. However, the DFT matrix has a special structure: its entries are powers of the primitive $N$th root of unity. This allows the DFT to be computed recursively by dividing the problem into smaller DFTs of size $N/2$ (for $N$ a power of $2$), combining the results using the so-called "butterfly" operations.
This recursive approach is the basis of the Fast Fourier Transform (FFT) algorithm, which reduces the computational complexity to $O(N \log N)$. The FFT exploits the symmetry and periodicity of the complex exponentials, making it much faster than a naive matrix multiplication for large $N$.