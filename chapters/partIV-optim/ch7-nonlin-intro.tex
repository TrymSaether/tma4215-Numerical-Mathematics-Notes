\section*{Introduction}

\emph{Non-linear systems} arise frequently in scientific computing, engineering, and applied mathematics.
Unlike linear systems, nonlinear systems can exhibit complex behavior such as \textbf{multiple solutions}, \textbf{bifurcations}, and \textbf{sensitivity to initial conditions}.
Solving nonlinear equations is therefore a central challenge in numerical mathematics.

\textbf{Optimization} is closely related to the study of nonlinear systems. Many optimization problems, especially those involving nonlinear objective functions or constraints, require solving nonlinear equations as part of their solution process. For example, finding the \emph{stationary points} of a function---where its gradient vanishes---leads directly to solving a system of nonlinear equations. Conversely, techniques developed for optimization, such as \emph{Newton's method} and its variants, are often employed to solve general nonlinear systems.

This part introduces \textbf{fundamental concepts} and \textbf{numerical methods} for analyzing and solving nonlinear systems and optimization problems, highlighting their deep interconnections and practical applications.

\begin{figure}[htbp]
  \centering
  \includestandalone[width=0.8\textwidth]{figures/nonlinear-stationary}
  \caption{A nonlinear function with multiple stationary points illustrates the complexity of nonlinear systems and optimization.}
  \label{fig:nonlinear-example}
\end{figure}
