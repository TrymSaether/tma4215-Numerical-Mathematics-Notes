% !TeX program = lualatex
% Standalone 3D plot: Jacobi polynomials P_n^{(\alpha,\beta)}(x) for n=0..4, \alpha,\beta in [-0.5,2]

\documentclass[tikz, border=5pt]{standalone}
\usepackage{pgfplots}
\pgfplotsset{compat=newest}
\usepackage{pgfmath}
\usepackage{amsmath}
\begin{document}

% Jacobi polynomial definition for small n
\pgfmathdeclarefunction{jacobiP}{4}{%
  % n, alpha, beta, x
  \pgfmathparse{ifthenelse(#1==0,1,
    ifthenelse(#1==1,0.5*(2*#2+2+#3+#2+#3+2)*(#4-1)+#2+1,
      ifthenelse(#1==2,0.5*(#2+#3+2)*(#2+#3+3)*(#4^2-1)+(#2+#3+2)*(#2-#3)*(#4-1)+(#2+1)*(#2+2),0)
    )
  )}
}

\begin{tikzpicture}
  \begin{axis}[
    scale only axis,
    width=12cm,
    height=8cm,
    view={-40}{15},
    domain=-2:2,
    samples=40,
    samples y=20,
    xlabel={$x$}, ylabel={$\alpha$}, zlabel={$P_2^{(\alpha,0)}(x)$},
    colormap/cool,
    colorbar,
    colorbar style={height=5.25cm, width=0.3cm, at={(1.05,0.49)}, anchor=west},
    title={Jacobi polynomial $P_2^{(\alpha,0)}(x)$ as a function of $x$ and $\alpha$},
    grid=major,
  ]
    % Example: n=2, beta=0
    \addplot3[surf,shader=interp,opacity=0.9] ({x},{y},{jacobiP(2,y,0,x)});
  \end{axis}
\end{tikzpicture}

\end{document}
