\documentclass[a4paper,10pt,landscape]{article}
% =============================
% PREAMBLE — LuaLaTeX (lean, NO/only)
% =============================

% Quiet harmless warnings (optional)
\usepackage{silence}
\WarningFilter{latex}{Command \showhyphens has changed}
\WarningFilter{latexfont}{Font shape}

% --- Language & fonts ---
\usepackage[english,shorthands=off]{babel}
\usepackage{fontspec}
\usepackage{amsmath,amsthm,mathtools}
\mathtoolsset{showonlyrefs,centercolon}
\usepackage[mathrm=sym,  warnings-off={mathtools-colon, mathtools-overbracket}]{unicode-math}

% Text fonts
\setmainfont{Fira Sans}
\setsansfont{IBM Plex Sans Condensed}
\setmonofont{Fira Code}
\setmathfont{Fira Math}
\setmathfont{STIX Two Math}[range={\mathscr,\mathbb,\mathfrak,\mathcal,\setminus,\Re,\Im,\triangleright,\top,\ddots,\vdots,\star,\bigcup},Scale=MatchUppercase]

% STIX fonts
% \setmainfont{STIX Two Text}
% \setsansfont{STIX Two Text}
% \setmonofont{STIX Two Text}
% \setmathfont{STIX Two Math}
% \setmathfont{STIX Two Math}[range={\mathscr,\mathbb,\mathfrak,\mathcal,\setminus,\Re,\Im,\triangleright,\top,\ddots,\vdots,\star,\bigcup},Scale=MatchUppercase]

% % Handwritten Font
% \setmainfont{Caveat}
% \setsansfont{Caveat}
% \setmonofont{Caveat}
% \setmathfont{Stix Two Math}

% --- Typography & lists ---
\usepackage[final]{microtype}
\UseMicrotypeSet[protrusion]{basicmath}
\microtypesetup{tracking=true,expansion=true}
\usepackage[table]{xcolor} % table colors without separate colortbl
\usepackage{enumitem}
\setlist{nosep}
\usepackage[autostyle]{csquotes}

% --- Structure & graphics ---
\usepackage{subfiles}                    % fast per-chapter compilation
\usepackage{booktabs,tabularx,array,multirow}
\usepackage{siunitx}
\sisetup{detect-all,group-minimum-digits=4,output-decimal-marker=.}
\usepackage{float}
\usepackage{subcaption}

% --- Layout, colors, boxes ---
% ==============================
% COLORS
% ==============================

% --- Semantic environment colors  ---
\definecolor{def-color}{HTML}{E40125}   % red (NTNU red)
\definecolor{thm-color}{HTML}{1346E4}   % deep blue
\definecolor{lem-color}{HTML}{05C4D9}   % cyan
\definecolor{exm-color}{HTML}{21340A}   % dark olive
\definecolor{rem-color}{HTML}{18B640}   % green
\definecolor{prop-color}{HTML}{9C27B0}  % purple
\definecolor{cor-color}{HTML}{FF5722}   % orange
\definecolor{alg-color}{HTML}{37474F}   % blue-grey
\definecolor{mot-color}{HTML}{9C27B0}   % reuse purple

% --- Utility accents (optional) ---
\definecolor{filing-color}{HTML}{607D8B}
\definecolor{railing-color}{HTML}{795548}
\definecolor{flag-color}{HTML}{009688}
\definecolor{spaceblack}{HTML}{403E3D}

% --- Soft yes/no backgrounds ---
\definecolor{yescolor}{RGB}{200,255,200}
\definecolor{nocolor}{RGB}{255,200,200}
\definecolor{headerblue}{RGB}{225,238,255}

% --- NTNU palette (reference only; prefer semantic names above) ---
\definecolor{ntnu-blue}{HTML}{00509E}
\definecolor{ntnu-red}{HTML}{E40125}
\definecolor{ntnu-green}{HTML}{BCD025}
\definecolor{ntnu-lightblue}{HTML}{6096D0}
\definecolor{ntnu-orange}{HTML}{EF8114}
\definecolor{ntnu-sand}{HTML}{CFB887}
\definecolor{ntnu-magenta}{HTML}{B01B81}
\definecolor{ntnu-yellow}{HTML}{F7D019}
\definecolor{ntnu-purple}{HTML}{482776}
\definecolor{ntnu-turquoise}{HTML}{3CBFBE}

% --- State colors (kept minimal; avoid near-duplicates) ---
\definecolor{state-yellow}{HTML}{F7D019}
\definecolor{state-orange}{HTML}{FF6F00}
\definecolor{state-gray}{HTML}{B0BEC5}
\definecolor{state-green}{HTML}{18B640}
\colorlet{state-blue}{thm-color!75}
\colorlet{state-red}{red!80}

% --- Hyperref colors (hooked in preamble's \hypersetup) ---
\definecolor{Link}{HTML}{1346E4} % align with thm-color (consistent brand)
\definecolor{Cite}{HTML}{0F766E} % teal 700 (better contrast)
\definecolor{URL}{HTML}{0E7490}  % cyan 700
\definecolor{File}{HTML}{B45309} % amber 700

% --- Auto tints (use these for colback; full colors for colframe) ---
\colorlet{def-tint}{def-color!8}
\colorlet{thm-tint}{thm-color!8}
\colorlet{lem-tint}{lem-color!10}
\colorlet{exm-tint}{exm-color!6}
\colorlet{rem-tint}{rem-color!10}
\colorlet{prop-tint}{prop-color!8}
\colorlet{cor-tint}{cor-color!8}
\colorlet{alg-tint}{alg-color!8}
\colorlet{mot-tint}{mot-color!8}
\colorlet{ntnu-yellow-tint}{ntnu-yellow!12}
\usepackage[most]{tcolorbox}
\usepackage{algorithm}
\usepackage[noend]{algpseudocode}
\usepackage[headheight=15pt,margin=2.5cm,includeheadfoot]{geometry}
\usepackage{parskip}

\usepackage{fancyhdr}
\pagestyle{fancy}
\fancyhf{}
\fancyhead[L]{\nouppercase{\leftmark}}
\fancyhead[R]{\thepage}
\renewcommand{\headrulewidth}{0.4pt}
\renewcommand{\footrulewidth}{0pt}

% --- Links & refs (keep order) ---
\usepackage[colorlinks,plainpages=false]{hyperref}
\hypersetup{
    colorlinks=true,
    linkcolor=thm-color,
    citecolor=thm-color,
    urlcolor=cyan,
    filecolor=magenta
}
\usepackage[capitalize,nameinlink,noabbrev]{cleveref}

% --- Your split macros (formatting vs notation) ---
% ==============================
% COMMANDS — formatting & convenience (non-semantic)
% ==============================

% Mathtools paired delimiters (with starred auto-size)
\DeclarePairedDelimiter{\abs}{\lvert}{\rvert}
\DeclarePairedDelimiter{\norm}{\lVert}{\rVert}
\DeclarePairedDelimiter{\set}{\lbrace}{\rbrace}

% Inner product with comma spacing
\DeclarePairedDelimiterX{\inner}[2]{\langle}{\rangle}{#1,\,#2}

% Derivatives (clear, flexible)
\NewDocumentCommand{\dv}{o m}{\frac{d\,#1}{d #2}}            % \dv[f]{x}
\NewDocumentCommand{\dvn}{m o m}{\frac{d^{#1}\,#2}{d #3^{#1}}} % \dvn{2}[f]{x}
\NewDocumentCommand{\pdv}{o m}{\frac{\partial\,#1}{\partial #2}}
\NewDocumentCommand{\pdvn}{m o m}{\frac{\partial^{#1}\,#2}{\partial #3^{#1}}}

% Common text shortcuts (optional; comment out if you prefer)
\newcommand{\ie}{i.e.\ }
\newcommand{\eg}{e.g.\ }
\newcommand{\aka}{aka.\ }
\newcommand{\st}{\text{s.t.}\ }   % formatting/convenience macros
% ==============================
% NOTATION — semantic symbols & operators
% ==============================

% Sets & fields
\newcommand{\N}{\mathbb{N}}
\newcommand{\Z}{\mathbb{Z}}
\newcommand{\Q}{\mathbb{Q}}
\newcommand{\R}{\mathbb{R}}
\newcommand{\C}{\mathbb{C}}
\newcommand{\F}{\mathbb{F}}

% Linear algebra operators
\DeclareMathOperator{\diag}{diag}
\DeclareMathOperator{\tr}{tr}
\DeclareMathOperator{\rank}{rank}
\DeclareMathOperator{\adj}{adj}
\DeclareMathOperator{\col}{col}
\DeclareMathOperator{\row}{row}
\DeclareMathOperator{\range}{range}
\DeclareMathOperator{\Span}{span}
\DeclareMathOperator{\Ker}{ker}
\DeclareMathOperator{\Img}{im}

% Optimization
\DeclareMathOperator*{\argmin}{arg\,min}
\DeclareMathOperator*{\argmax}{arg\,max}

% Probability/statistics (optional)
\DeclareMathOperator{\Var}{Var}
\DeclareMathOperator{\Cov}{Cov}
\newcommand{\E}{\mathbb{E}}

% Numbering
% \numberwithin{equation}{chapter}
   % semantic symbols/sets/operators

% --- Theorem environments ---
\tcbset{
    base theorem style/.style={
            enhanced, breakable, sharp corners,
            fonttitle=\bfseries,
            before skip=10pt, after skip=10pt,
            separator sign=.
        },
}

\newtcbtheorem[number within=chapter]{definition}{Definition}{%
    enhanced jigsaw,
    colback=def-color!10, colframe=def-color!80!black, coltitle=black,
    fonttitle=\bfseries,
    attach boxed title to top left={xshift=10pt,yshift=-\tcboxedtitleheight/2},
    boxed title style={colback=def-color!10,colframe=def-color!80!black,height=16pt,valign=center,bean arc},
    label separator={}, top=8pt,bottom=2pt,left=4pt,right=4pt,boxrule=1pt
}{def:}

\newtcbtheorem[use counter from=definition]{theorem}{Theorem}{%
    enhanced jigsaw,
    colback=thm-color!10, colframe=thm-color!80!black, coltitle=black,
    fonttitle=\bfseries,
    attach boxed title to top left={xshift=10pt,yshift=-\tcboxedtitleheight/2},
    boxed title style={colback=thm-color!10,colframe=thm-color!80!black,height=16pt,valign=center,bean arc},
    label separator={}, top=8pt,bottom=2pt,left=4pt,right=4pt,boxrule=1pt
}{thm:}

\newtcbtheorem{lemma}{Lemma}{%
    enhanced jigsaw,
    frame hidden, boxrule=0pt,
    colback=lem-color!10, colframe=lem-color!10,
    fonttitle=\sffamily\bfseries,
    attach boxed title to top left={yshift=-\tcboxedtitleheight},
    boxed title style={boxrule=0pt,boxsep=2pt,interior code={\fill[lem-color, smooth]
                    (interior.north west)--(interior.south west)--([xshift=-2mm]interior.south east)--
                    ([xshift=2mm]interior.north east)--cycle;}},
    label separator={}, borderline north={1pt}{0pt}{lem-color},
    before upper={\hspace{\tcboxedtitlewidth}},
    sharp corners, top=2pt,bottom=2pt,left=4pt,right=4pt
}{lem}

\newtcbtheorem{example}{Example}{%
    base theorem style,
    colback=white, colframe=white,
    borderline west={1.5pt}{0pt}{exm-color},
    left=8pt, coltitle=exm-color, boxsep=2pt, top=0pt, bottom=0pt
}{ex}

\NewTColorBox{solution}{ O{} }{%
    enhanced, colback=white, colframe=white,
    borderline east={1.5pt}{0pt}{rem-color},
    left=8pt,right=4pt,top=4pt,bottom=4pt,
    before skip=10pt,after skip=10pt,breakable,sharp corners,boxrule=0pt,frame hidden,parbox=false,
    overlay={\node[rotate=-90, anchor=south west,font=\tiny\bfseries]
            at ([yshift=3pt, xshift=-2pt]frame.north east) {\MakeUppercase{Løsning}};}
}

\newtcbtheorem{remark}{Remark}{%
    base theorem style, colback=white, colframe=white,
    borderline west={1.5pt}{0pt}{rem-color},
    left=8pt, top=0pt, bottom=0pt, boxsep=2pt, coltitle=black
}{rem}

\newtcbtheorem{proposition}{Proposition}{%
    enhanced jigsaw,
    colback=prop-color!10, colframe=prop-color!80!black,
    fonttitle=\bfseries,
    attach boxed title to top left={yshift=-\tcboxedtitleheight},
    boxed title style={boxrule=0pt,boxsep=2.5pt,colback=prop-color!80!black,
            colframe=prop-color!80!black,sharp corners=uphill},
    label separator={}, top=\tcboxedtitleheight,bottom=2pt,left=2pt,right=2pt,
    before skip=10pt,after skip=10pt,breakable
}{prop}

\newtcbtheorem{corollary}{Corollary}{%
    enhanced jigsaw,
    colback=cor-color!10, colframe=cor-color!80!black, boxrule=0pt,
    fonttitle=\sffamily\bfseries, coltitle=black,
    label separator={},
    description font=\normalfont\sffamily, description delimiters={(}{)},
    attach title to upper, after title={.\ },
    frame hidden, borderline west={2pt}{0pt}{cor-color},
    sharp corners, top=2pt,bottom=2pt,left=5pt,right=5pt,
    before skip=10pt,after skip=10pt,breakable
}{cor}

\newtcbtheorem{summary}{Summary}{%
    enhanced jigsaw,
    colback=ntnu-yellow!10, colframe=ntnu-yellow!80!black,
    coltitle=black, fonttitle=\bfseries,
    attach boxed title to top left={xshift=10pt,yshift=-\tcboxedtitleheight/2},
    boxed title style={colback=ntnu-yellow!10,colframe=ntnu-yellow!80!black,height=16pt,bean arc},
    label separator={}, sharp corners, top=6pt,bottom=2pt,left=2pt,right=2pt,
    before skip=10pt,after skip=10pt,breakable
}{sum}

% --- Property environment ---
\newtcbtheorem{property}{Property}{%
    enhanced jigsaw,
    colback=alg-color!10, colframe=alg-color!80!black,
    fonttitle=\bfseries,
    attach boxed title to top left={yshift=-\tcboxedtitleheight/2},
    boxed title style={boxrule=0pt,boxsep=2pt,colback=alg-color!80!black,
            colframe=alg-color!80!black,sharp corners=uphill,height=16pt,valign=center},
    label separator={}, top=8pt,bottom=2pt,left=4pt,right=4pt,
    before skip=10pt,after skip=10pt,breakable
}{prop:}

\renewenvironment{proof}[1][\proofname]{%
    \begin{tcolorbox}[blanker,borderline west={2pt}{0pt}{thm-color},
            before skip=10pt, after skip=10pt, left=8pt, breakable]%
        \textbf{#1}. }{\,\hfill\qedsymbol\end{tcolorbox}}


% --- TikZ / pgfplots ---
\usepackage{standalone}
\usepackage{pgfplots,tikz}
\pgfplotsset{
    compat=newest,
    tick label style={/pgf/number format/fixed},
    scaled ticks=false
}
\usetikzlibrary{
    arrows.meta,calc,positioning,backgrounds,patterns,decorations.pathreplacing,
    3d, matrix, shapes.geometric
}
\usepgfplotslibrary{groupplots,colormaps}
\tikzset{>=Latex}

\usepackage{multicol}
% Set up three columns with small separation and a line between columns
\setlength{\columnsep}{0.10cm}
\setlength{\columnseprule}{0.2pt}
% Page geometry for cheat sheet

\geometry{top=-0.75cm, bottom=-0.75cm, left=0.1cm, right=0.1cm,heightrounded,includeheadfoot}
\pagestyle{empty}
% No headers or footers
\raggedbottom
\raggedright
% Remove header and footer lines
\usepackage{fancyhdr}
\fancyhf{}
\renewcommand{\headrulewidth}{0pt}
\renewcommand{\footrulewidth}{0pt}
% section colors (red, blue, light blue)
\usepackage{xcolor}
\definecolor{sectioncolor}{RGB}{139,0,0} % Dark red
\definecolor{subsectioncolor}{RGB}{0,0,139} % Dark blue
\definecolor{subsubsectioncolor}{RGB}{0,102,204} % Medium blu

% Section title smaller, colored fonts

\usepackage{sectsty}
\sectionfont{\color{sectioncolor}\small\bfseries}
\subsectionfont{\color{subsectioncolor}\footnotesize}
\subsubsectionfont{\color{subsubsectioncolor}\scriptsize}

% Add line before section titles, keeping the same colors, and reduce space
\usepackage{titlesec}
\titleformat{\section}
  {\color{sectioncolor}\normalfont\small\bfseries}
  {\thesection}{1em}{\color{sectioncolor}\rule{\linewidth}{0.4pt}\vspace{0.5ex}\\}
\titleformat{\subsection}
  {\color{subsectioncolor}\normalfont\footnotesize\bfseries}
  {\thesubsection}{1em}{\color{subsectioncolor}\rule{\linewidth}{0.3pt}\vspace{0.4ex}\\}
\titleformat{\subsubsection}
  {\color{subsubsectioncolor}\normalfont\scriptsize\bfseries}
  {\thesubsubsection}{1em}{\color{subsubsectioncolor}\rule{\linewidth}{0.2pt}\vspace{0.3ex}\\}



% Reduce space before and after section titles
\usepackage{titlesec}
\titlespacing*{\section}{0pt}{0.6ex plus .2ex minus .2ex}{0.5ex plus .2ex}
% Reduce space before and after subsection titles
\titlespacing*{\subsection}{0pt}{0.4ex plus .2ex minus .2ex}{0.3ex plus .2ex}
\titlespacing*{\subsubsection}{0pt}{0.2ex plus .2ex minus .2ex}{0.2ex plus .2ex}

% paragraph spacing and smaller
\setlength{\parskip}{0pt}
\setlength{\parindent}{0pt}

% Reduce space between items in itemize
\usepackage{enumitem}
\setitemize{noitemsep, topsep=0pt, leftmargin=*, parsep=0pt, partopsep=0pt}

% Reduce space between lines
\usepackage{setspace}
\setstretch{0.10}

% Reduce space around equations
\setlength{\abovedisplayskip}{1pt}
\setlength{\belowdisplayskip}{1pt}
\setlength{\abovedisplayshortskip}{1pt}
\setlength{\belowdisplayshortskip}{1pt}


\begin{document}
\begin{multicols*}{3}
    \scriptsize
    \section*{Linear Algebra}
    \begin{description}
        \item[Vector norms:] $\norm{\mathbf{x}}_1 = \sum_i |x_i|$, $\norm{\mathbf{x}}_2 = \sqrt{\sum_i x_i^2}$, $\norm{\mathbf{x}}_\infty = \max_i |x_i|$
        \item[Matrix norms:] $\norm{A}_1 = \max_j \sum_i |a_{ij}|$ (col sum), $\norm{A}_\infty = \max_i \sum_j |a_{ij}|$ (row sum), $\norm{A}_2 = \sqrt{\lambda_{\max}(A^TA)}$, $\norm{A}_F = \sqrt{\sum_{i,j} a_{ij}^2}$ (Frobenius)
        \item[Eigenvalues:] $A\mathbf{v} = \lambda\mathbf{v}$. Characteristic poly: $\det(A-\lambda I) = 0$
        \item[SVD:] $A = U\Sigma V^T$ with $U,V$ orthogonal, $\Sigma$ diagonal. $\sigma_{\max} = \norm{A}_2$, $\kappa_2(A) = \sigma_{\max}/\sigma_{\min}$
        \item[Spectral radius:] $\rho(A) = \max_i |\lambda_i| \leq \norm{A}$ for any matrix norm
        \item[Strictly diagonally dominant (SDD):] $|a_{ii}| > \sum_{j \neq i} |a_{ij}|$ for all $i$. Guarantees non-singular matrix and convergence of Jacobi/Gauss-Seidel
        \item[Matrix classes:] SPD (symmetric positive definite): $A = A^T$ and $\mathbf{x}^T A \mathbf{x} > 0$ for $\mathbf{x} \neq \mathbf{0}$. Orthogonal: $Q^TQ = I$. Unitary: $U^*U = I$
        \item[Rank:] $\text{rank}(A) = $ number of linearly independent rows/cols = number of nonzero singular values
        \item[Determinant properties:] $\det(AB) = \det(A)\det(B)$, $\det(A^T) = \det(A)$, $\det(A^{-1}) = 1/\det(A)$
        \item[Trace:] $\text{tr}(A) = \sum_i a_{ii} = \sum_i \lambda_i$. $\text{tr}(AB) = \text{tr}(BA)$
        \item[Matrix inverses:] $(AB)^{-1} = B^{-1}A^{-1}$, $(A^T)^{-1} = (A^{-1})^T$. For SPD: $A^{-1}$ is also SPD
    \end{description}

    \section*{Non-linear Equations \& Fixed-Point}
    \begin{description}
        \item[Banach FPT:] A self map $g$ on a complete metric space $X$ is a contraction if $\exists\, 0<\kappa<1$ such that $\abs{g(x)-g(y)} \leq \kappa\abs{x-y}$ for all $x,y$. Then $\mathbf{x}^{(k+1)}=g(\mathbf{x}^{(k)})$ converges to unique fixed point. Error: $\abs{\mathbf{x}^{(k)}-\mathbf{x}^*} \leq \frac{\kappa^k}{1-\kappa}\abs{\mathbf{x}^{(1)}-\mathbf{x}^{(0)}}$
        \item[General FP:] Starting with $\mathbf{x}^{(0)}$, generate $\mathbf{x}^{(k+1)} = g(\mathbf{x}^{(k)})$. Convergence requires $g$ to be a contraction on a region containing the fixed point.
        \item[Newton's method:] For $f(x)=0$ with $f: \R^n\to\R^n$: $\mathbf{x}^{(k+1)} = \mathbf{x}^{(k)} - J_f(\mathbf{x}^{(k)})^{-1}f(\mathbf{x}^{(k)})$. Scalar: $x_{k+1} = x_k - \frac{f(x_k)}{f'(x_k)}$. Locally quadratically convergent.
        \item[Order $p$:] $\lim_{k\to\infty} \frac{\abs{\mathbf{x}^{(k+1)}-\mathbf{x}^*}}{\abs{\mathbf{x}^{(k)}-\mathbf{x}^*}^p} = C \neq 0$
        \item[Conditioning:] For simple root $\mathbf{x}^*$ of $f$, relative condition number $\approx \frac{1}{\abs{\mathbf{x}^* f'(\mathbf{x}^*)}}$
    \end{description}

    \section*{Linear Systems}
    \subsection*{Direct Methods}
    \begin{description}
        \item[Gaussian elimination:] $O(n^3/3)$ flops for $n \times n$ system. Forward elimination + back substitution
        \item[Elimination step:] For pivot row $k$: $l_{j,k} = a_{j,k}/a_{k,k}$ (requires $a_{k,k} \neq 0$). Update: $a_{j,p} \leftarrow a_{j,p} - l_{j,k}a_{k,p}$, $b_j \leftarrow b_j - l_{j,k}b_k$ for $j>k$, $p \geq k+1$
        \item[Partial pivoting:] Choose largest $|a_{j,k}|$ in column $k$ below diagonal as pivot. Essential for numerical stability
        \item[LU factorization:] $PA = LU$ with permutation $P$, unit lower triangular $L$, upper triangular $U$. Solve $L\mathbf{y} = P\mathbf{b}$ (forward), then $U\mathbf{x} = \mathbf{y}$ (backward)
        \item[Cholesky:] For SPD matrix: $A = LL^T$ with $L$ lower triangular, positive diagonal. $O(n^3/6)$ flops, no pivoting needed. Fails if $A$ not SPD
        \item[QR decomposition:] $A = QR$ with $Q$ orthonormal, $R$ upper triangular. For $A\mathbf{x} = \mathbf{b}$: solve $R\mathbf{x} = Q^T\mathbf{b}$. Cost: $O(2n^3/3)$ flops
        \item[Householder reflectors:] $H = I - 2\mathbf{v}\mathbf{v}^T/\|\mathbf{v}\|_2^2$ where $\mathbf{v} = \mathbf{x} \pm \|\mathbf{x}\|_2\mathbf{e}_1$ (choose sign to avoid cancellation). More stable than Gram-Schmidt
        \item[Givens rotations:] Zero element $(j,i)$ using $G_{ij}(\theta)$ where $c = \cos\theta = \frac{a_{ii}}{\sqrt{a_{ii}^2+a_{ji}^2}}$, $s = \sin\theta = \frac{a_{ji}}{\sqrt{a_{ii}^2+a_{ji}^2}}$
        \item[Gram-Schmidt:] $\mathbf{q}_1 = \mathbf{a}_1/\|\mathbf{a}_1\|_2$. For $k \geq 2$: $\mathbf{u}_k = \mathbf{a}_k - \sum_{j=1}^{k-1}(\mathbf{a}_k^T\mathbf{q}_j)\mathbf{q}_j$, $\mathbf{q}_k = \mathbf{u}_k/\|\mathbf{u}_k\|_2$. Numerically unstable; prefer modified GS or Householder
        \item[Least squares:] Minimize $\|\mathbf{b} - A\mathbf{x}\|_2^2$. Normal equations: $A^TA\mathbf{x} = A^T\mathbf{b}$ (condition number $\kappa_2(A^TA) = \kappa_2(A)^2$). QR method: $R\mathbf{x} = Q^T\mathbf{b}$ (condition number $\kappa_2(A)$). Prefer QR for better stability
    \end{description}

    \subsection*{Conditioning \& Spectra}
    \begin{description}
        \item[Condition number:] $\kappa(A) = \norm{A}\norm{A^{-1}}$. Measures sensitivity of $x$ in $Ax=\mathbf{b}$
        \item[Gershgorin discs:] Each eigenvalue $\lambda \in S_j = \{z \in \mathbb{C}: \abs{z-a_{jj}} \leq \sum_{k\neq j}\abs{a_{jk}}\}$
    \end{description}

    \subsection*{Iterative Methods}
    \begin{description}
        \item[General form:]
              Stationary iteration for solving \(A\mathbf{x} = \mathbf{b}\):
              \[
                  \mathbf{x}^{(k+1)} = B\,\mathbf{x}^{(k)} + \mathbf{f},
              \]
              where \(B\) is the iteration matrix and \(\mathbf{f}\) is a constant vector.
              If \(\mathbf{x}^\ast\) is the exact solution and \(\mathbf{e}^{(k)} = \mathbf{x}^{(k)} - \mathbf{x}^\ast\),
              then
              \[
                  \mathbf{e}^{(k+1)} = B\,\mathbf{e}^{(k)} \quad\Rightarrow\quad
                  \mathbf{e}^{(k)} = B^k \mathbf{e}^{(0)}.
              \]

        \item[Convergence:]
              The iteration converges for every initial guess \(\mathbf{x}^{(0)}\) iff
              \[
                  \rho(B) < 1,
              \]
              where \(\rho(B)\) is the spectral radius of \(B\).

              Asymptotic error decay (in any norm compatible with \(B\)):
              \[
                  \|\mathbf{e}^{(k)}\| \le C\,\rho(B)^k \,\|\mathbf{e}^{(0)}\|
              \]
              for some constant \(C\) independent of \(k\).

        \item[Richardson:]
              For \(A\) (typically SPD) and relaxation parameter \(\omega>0\),
              \[
                  \mathbf{x}^{(k+1)} = \mathbf{x}^{(k)} + \omega\bigl(\mathbf{b} - A\mathbf{x}^{(k)}\bigr)
                  = (I - \omega A)\mathbf{x}^{(k)} + \omega\mathbf{b},
              \]
              so the iteration matrix is \(B_R = I - \omega A\).

              For \(A\) SPD with eigenvalues
              \(0 < \lambda_{\min} \le \lambda_{\max}\),
              \begin{itemize}
                  \item Convergence iff \(0 < \omega < \dfrac{2}{\lambda_{\max}}\).
                  \item Optimal parameter:
                        \[
                            \omega_{\text{opt}} = \frac{2}{\lambda_{\min} + \lambda_{\max}}.
                        \]
              \end{itemize}

        \item[Jacobi:]
              Let
              \[
                  A = D - (L + U),
              \]
              where \(D\) is the diagonal of \(A\), and \(L,U\) are the strictly lower/upper
              triangular parts (with this sign convention).

              Iteration matrix and update:
              \begin{align*}
                  B_J                & = D^{-1}(L + U),                            \\
                  \mathbf{x}^{(k+1)} & = B_J\,\mathbf{x}^{(k)} + D^{-1}\mathbf{b}, \\
                  x_i^{(k+1)}        & = \frac{1}{a_{ii}}
                  \left[b_i - \sum_{j \ne i} a_{ij}\,x_j^{(k)}\right].
              \end{align*}

              \textbf{Convergence:}
              \begin{itemize}
                  \item If \(A\) is strictly diagonally dominant (SDD) by rows, then Jacobi converges.
                  \item More generally, Jacobi converges iff \(\rho(B_J) < 1\).
              \end{itemize}

        \item[Gauss--Seidel:]
              With the same splitting \(A = D - (L+U)\), Gauss--Seidel corresponds to
              \[
                  (D - L)\,\mathbf{x}^{(k+1)} = U\,\mathbf{x}^{(k)} + \mathbf{b},
              \]
              so the iteration matrix is
              \[
                  B_{GS} = (D - L)^{-1} U.
              \]
              Componentwise:
              \[
                  x_i^{(k+1)} = \frac{1}{a_{ii}}\left[
                      b_i
                      - \sum_{j<i} a_{ij}\,x_j^{(k+1)}
                      - \sum_{j>i} a_{ij}\,x_j^{(k)}
                      \right].
              \]

              \textbf{Convergence:}
              \begin{itemize}
                  \item If \(A\) is SPD, Gauss--Seidel is convergent (monotone in the \(A\)-norm).
              \end{itemize}

        \item[SOR (Successive Over-Relaxation):]
              For \(\omega \in \mathbb{R}\setminus\{0\}\),
              \[
                  x_i^{(k+1)} =
                  (1-\omega)\,x_i^{(k)}
                  + \frac{\omega}{a_{ii}}\left[
                      b_i
                      - \sum_{j<i} a_{ij}\,x_j^{(k+1)}
                      - \sum_{j>i} a_{ij}\,x_j^{(k)}
                      \right].
              \]
              In matrix form (again with \(A = D - (L+U)\)):
              \[
                  (D - \omega L)\,\mathbf{x}^{(k+1)}
                  = \bigl[(1-\omega)D + \omega U\bigr]\mathbf{x}^{(k)} + \omega\mathbf{b},
              \]
              hence
              \[
                  B_{\omega} = (D - \omega L)^{-1}\bigl[(1-\omega)D + \omega U\bigr].
              \]

              \textbf{Convergence (classical results):}
              \begin{itemize}
                  \item For \(A\) SPD (Ostrowski):
                        SOR converges iff \(0 < \omega < 2\).
                  \item If \(A\) is SDD by rows, SOR converges for \(0<\omega\le 1\).
                  \item Under additional assumptions (e.g. \(A\) has the A-property and
                        \(\rho(B_J) < 1\)), the optimal relaxation parameter is
                        \[
                            \omega_{\text{opt}} =
                            \frac{2}{1 + \sqrt{1 - \rho(B_J)^2}}.
                        \]
              \end{itemize}

        \item[Convergence rates (comparison):]
              For matrices with the A-property (in particular for many SPD
              tridiagonal problems),
              \[
                  \rho(B_{GS}) = \rho(B_J)^2
                  \quad\text{and}\quad
                  \rho\bigl(B_{\omega_{\text{opt}}}\bigr)
                  < \rho(B_{GS}) = \rho(B_J)^2 < \rho(B_J),
              \]
              so SOR with \(\omega_{\text{opt}}\) is fastest, followed by Gauss--Seidel, then Jacobi.
    \end{description}


    % --- Column 2 ---
    \section*{Interpolation \& Splines}
    \subsection*{Lagrange \& Newton}
    \begin{description}
        \item[Lagrange form:] $P_n(x) = \sum_{j=0}^n f(x_j)L_j(x)$, $L_j(x) = \prod_{i\neq j}\frac{x-x_i}{x_j-x_i}$
        \item[Newton form:] Uses divided differences $f[x_0,x_1,\ldots,x_k]$ recursively: $P_n(x) = f[x_0] + f[x_0,x_1](x-x_0) + \ldots + f[x_0,\ldots,x_n]\prod_{j=0}^{n-1}(x-x_j)$
        \item[Divided differences:] $f[x_i] = f(x_i)$, $f[x_i,x_{i+1}] = \frac{f[x_{i+1}]-f[x_i]}{x_{i+1}-x_i}$, $f[x_i,\ldots,x_{i+k}] = \frac{f[x_{i+1},\ldots,x_{i+k}] - f[x_i,\ldots,x_{i+k-1}]}{x_{i+k}-x_i}$
        \item[Error bound:] For $f \in C^{n+1}[a,b]$ and distinct nodes $x_0,\ldots,x_n$:
              \[f(x) - P_n(x) = \frac{f^{(n+1)}(\xi)}{(n+1)!}\prod_{j=0}^n(x-x_j)\]
              % Chebyshev nodes $x_j = \cos\left(\frac{(2j+1)\pi}{2(n+1)}\right)$ minimize $\max_{x \in [-1,1]}\left|\prod_{j=0}^n(x-x_j)\right|$
              % \item[Runge phenomenon:] High-degree polynomial interpolation with equispaced nodes can exhibit large oscillations. Use Chebyshev nodes or splines instead
    \end{description}

    \subsection*{Splines}
    \begin{description}
        \item[Definition.] A (univariate $s: \R \to \R$) spline of degree $k\ge 1$ on the partition $a=x_0<\cdots<x_n=b$ is
              \[
                  \mathcal{S}_k := \left\{\, s\in C^{k-1}[a,b] \;:\; s|_{[x_i,x_{i+1}]} \in \mathbb{P}_k \text{ for } i=0,\ldots,n-1 \,\right\},
              \]
              $\mathbb{P}_k$ denotes the polynomials $s(x) = p_i(x)$ of $\deg(p_i) \le k$.

        \item[Dimension.]
              \[
                  \dim(\mathcal{S}_k) = n(k+1) - k(n-1) = n + k.
              \]
              (There are $n$ pieces, each with $k+1$ coefficients, and $k$ continuity constraints at each of the $n-1$ interior knots.)

        \item[Knot vector:] $\Delta = \{x_0, x_1, \ldots, x_n\}$ with $n+1$ knots defining $n$ subintervals
        \item[Error bound.] If $f\in C^{k+1}[a,b]$ and $s\in\mathcal{S}_k$ is an interpolatory spline (made unique by adding $k-1$ extra conditions), then on a quasi-uniform mesh
              \[
                  \|f - s\|_\infty \le C\, h^{k+1}\,\|f^{(k+1)}\|_\infty, \quad h=\max_i h_i, \quad h_i=x_{i+1}-x_i
              \]
              for a const. $C$ indep. of $h$ (but dep. on $k$).
        \item[Piecewise linear ($k=1$):] For $s_1 \in C^0$, $x\in[x_i,x_{i+1}]$,
              \[
                  s_1(x)= y_i + \frac{y_{i+1}-y_i}{h_i}\,(x-x_i),
                  \qquad \|f-s_1\|_\infty = O(h^2).
              \]
        \item[Piecewise quadratic ($k=2$).] $s_2\in C^1$ and, with one additional condition beyond nodal interpolation (since $k-1=1$),
              \[
                  \|f-s_2\|_\infty = O(h^3).
              \]
    \end{description}

    \subsection*{Cubic Splines}
    \begin{description}
        \item[Definition:] $s \in C^{2}[a,b]$, $s|_{[x_i,x_{i+1}]} \in \mathbb{P}_3$ on $a=x_0 < x_1 < \ldots < x_n=b$
        \item[Error bound:] For $f \in C^4[a,b]$: $\norm{f - s}_\infty \leq \frac{5h^4}{384}\norm{f^{(4)}}_\infty$ where $h = \max_i h_i$
        \item[Second derivative form:] Let $h_i=x_{i+1}-x_i$, $z_i=s''(x_i)$:
              \[s_i(x) = \frac{z_{i+1}}{6h_i}(x-x_i)^3 + \frac{z_i}{6h_i}(x_{i+1}-x)^3 + \left(\frac{y_{i+1}}{h_i} - \frac{z_{i+1}h_i}{6}\right)(x-x_i) + \left(\frac{y_i}{h_i} - \frac{z_ih_i}{6}\right)(x_{i+1}-x)\]
        \item[Tridiagonal system:] For $i=1,\ldots,n-1$:
              \[h_{i-1}z_{i-1} + 2(h_{i-1}+h_i)z_i + h_iz_{i+1} = 6\left[\frac{y_{i+1}-y_i}{h_i} - \frac{y_i-y_{i-1}}{h_{i-1}}\right]\]
        \item[Boundary conditions:]
              \begin{itemize}
                  \item Natural: $z_0=z_n=0$ ($s''(a)=s''(b)=0$)
                  \item Clamped: $s'(a), s'(b)$ specified
                  \item Not-a-knot: $s'''$ continuous at $x_1,x_{n-1}$
                  \item Periodic: $s(a)=s(b)$, $s'(a)=s'(b)$, $s''(a)=s''(b)$
              \end{itemize}
        \item[Variational property:] Natural cubic spline minimizes $\int_a^b [g''(x)]^2 dx$ among all $g \in C^2[a,b]$ interpolating the data
        \item[Optimality:] Cubic splines achieve optimal approximation order $O(h^4)$ for $C^4$ functions
    \end{description}

    \subsection*{B-splines}
    \begin{description}
        \item[Basis:] $\{B_{i,k+1}\}$ forms basis for spline space $\mathcal{S}_k$ of degree $k$ splines
        \item[Support:] $B_{i,k+1}$ has support on $[t_i,t_{i+k+1}]$ where $\Delta$ is extended knot sequence
        \item[Properties:] Partition of unity: $\sum_i B_{i,k+1}(x) = 1$; Non-negativity: $B_{i,k+1}(x) \geq 0$; Local support
        \item[Cox-de Boor recursion:] $B_{i,1}(x) = \begin{cases} 1 & \text{if } t_i \leq x < t_{i+1} \\ 0 & \text{otherwise} \end{cases}$
              \[B_{i,k+1}(x) = \frac{x-t_i}{t_{i+k}-t_i}B_{i,k}(x) + \frac{t_{i+k+1}-x}{t_{i+k+1}-t_{i+1}}B_{i+1,k}(x)\]
        \item[Representation:] Any spline $s(x) = \sum_i c_i B_{i,k+1}(x)$ with control points $c_i$
        \item[Knot insertion:] Adding knots refines spline without changing its shape, enabling adaptive approximation
    \end{description}


    \section*{Orthogonal Polynomials \& Fourier}
    \subsection*{Orthogonal Polynomials}
    \begin{description}
        \item[Inner product:] $(f,g)_w = \int_a^b f(x)g(x)w(x)dx$ with weight $w(x) > 0$
        \item[Gram-Schmidt:] Orthogonalize $\{1,x,x^2,\ldots\}$ to get $\{p_0,p_1,p_2,\ldots\}$
        \item[Three-term recurrence:] $p_{k+1}(x) = (a_k x + b_k)p_k(x) - c_k p_{k-1}(x)$
        \item[Chebyshev polynomials:] $T_k(x) = \cos(k\arccos x)$, $w(x) = 1/\sqrt{1-x^2}$, $[-1,1]$. $T_{k+1} = 2xT_k - T_{k-1}$, $|T_k(x)| \leq 1$
        \item[Legendre polynomials:] $w(x) = 1$ on $[-1,1]$. $(2k+1)P_{k+1} = (2k+1)xP_k - kP_{k-1}$
        \item[Zeros property:] Orthogonal polynomial $p_n$ has $n$ simple real zeros in $(a,b)$
    \end{description}

    \subsection*{Fourier Series}
    \begin{description}
        \item[Complex form:] $f(x) = \sum_{n=-\infty}^{\infty} c_n e^{in\pi x/L}$, $c_n = \frac{1}{2L}\int_{-L}^L f(x)e^{-in\pi x/L}dx$
        \item[Real form:] $f(x) = \frac{a_0}{2} + \sum_{n=1}^{\infty}[a_n\cos(n\pi x/L) + b_n\sin(n\pi x/L)]$
        \item[Parseval's theorem:] $\frac{1}{2L}\int_{-L}^L |f(x)|^2dx = \sum_{n=-\infty}^{\infty}|c_n|^2 = \frac{|a_0|^2}{2} + \frac{1}{2}\sum_{n=1}^{\infty}(|a_n|^2 + |b_n|^2)$
        \item[Gibbs phenomenon:] Fourier series exhibits ~9\% overshoot near discontinuities
        \item[Convergence:] Pointwise convergence if $f$ piecewise smooth; uniform convergence if $f$ continuous and periodic
    \end{description}

    \section*{Numerical Differentiation}
    \begin{description}
        \item[Forward difference:] $f'(x) \approx \frac{f(x+h)-f(x)}{h}$ (error $O(h)$)
        \item[Backward difference:] $f'(x) \approx \frac{f(x)-f(x-h)}{h}$
        \item[Central difference:] $f'(x) \approx \frac{f(x+h)-f(x-h)}{2h}$ (error $O(h^2)$)
        \item[Second derivative:] $f''(x) \approx \frac{f(x+h)-2f(x)+f(x-h)}{h^2}$ (error $O(h^2)$)
        \item[Optimal step size:] Balance truncation $\propto h^p$ and rounding $\propto h^{-1}$: $h \approx \epsilon^{1/(p+1)}$
    \end{description}

    % --- Column 3 ---
    \section*{Numerical Integration}
    \subsection*{Newton-Cotes}
    \begin{description}
        \item[\textbf{Trapezoidal:}] $T(a,b) = \frac{b-a}{2}[f(a)+f(b)]$. Error: $I-T = -\frac{(b-a)^3}{12}f''(\xi)$. Composite: error $-\frac{(b-a)h^2}{12}f''(\xi)$
        \item[\textbf{Simpson's:}]
              \[S(a,b) = \frac{b-a}{6}[f(a) + 4f(c) + f(b)], \quad c=\frac{a+b}{2}.\]
              Error: $-\frac{(b-a)^5}{2880}f^{(4)}(\xi)$
        \item[\textbf{Composite Simpson:}]
              \[\int_a^b f(x)\,dx \approx \frac{h}{3}\left[f(x_0) + 4\sum_{j	ext{ odd}}f(x_j) + 2\sum_{j	ext{ even}}f(x_j) + f(x_n)\right].\]
              Error: $-\frac{(b-a)h^4}{180}f^{(4)}(\xi)$
    \end{description}

    \subsection*{Richardson \& Romberg}
    \begin{description}
        \item[\textbf{Richardson extrapolation:}] If $Q(h) = I + \alpha_1 h + \alpha_2 h^2 + \ldots$, then $	ilde{Q}(h) = \frac{2Q(h/2) - Q(h)}{2-1}$ eliminates $h$ term
        \item[\textbf{Romberg integration:}] Recursive Richardson on trapezoidal rule. $\mathcal{A}_{k,j} = \frac{4^j \mathcal{A}_{k,j-1} - \mathcal{A}_{k-1,j-1}}{4^j - 1}$
        \item[\textbf{Accuracy:}] $\mathcal{A}_{k,k} = I + O(h_k^{2(k+1)})$ with super-geometric convergence
    \end{description}

    \subsection*{Gaussian Quadrature}
    \begin{description}
        \item[\textbf{Principle:}] Choose both nodes $x_i$ and weights $\omega_i$ optimally. $Q_n(f) = \sum_{i=1}^n \omega_i f(x_i)$ exact for polynomials of degree $\leq 2n-1$
        \item[\textbf{Gauss-Legendre:}] Nodes are zeros of Legendre polynomial $P_n(x)$ on $[-1,1]$
        \item[\textbf{Gauss-Chebyshev:}] Nodes are zeros of $T_n(x)$, weight $w(x) = 1/\sqrt{1-x^2}$, $\omega_i = \pi/n$
        \item[\textbf{Error:}] $\frac{f^{(2n)}(\xi)}{(2n)!}\int_a^b w(x)\omega_n(x)^2 dx$ where $\omega_n(x) = \prod_{i=1}^n(x-x_i)$
    \end{description}

    \subsection*{Adaptive \& Gauss}
    \begin{description}
        \item[\textbf{Adaptive Simpson:}] Apply Simpson on $[a,b]$ and halves $[a,c],[c,b]$. Error estimate: $\frac{S_2-S_1}{15}$. If $|S_2-S_1|$ below tolerance, accept $S_2$
        \item[\textbf{Gauss:}] Choose nodes $x_i$ as zeros of orthogonal polynomial $p_n$ w.r.t. weight $w$. $Q_{w,n}(f) = \sum_{i=1}^n \omega_i f(x_i)$ has exactness degree $2n-1$. Error: $\frac{f^{(2n)}(\xi)}{(2n)!}\int_a^b w(x)\omega_n(x)^2 dx$
    \end{description}
    \section*{ODEs}
    \subsection*{IVPs \& One-step Methods}
    \begin{description}
        \item[\textbf{IVP:}] $y'(t)=f(t,y(t))$, $y(t_0)=y_0$. If $f$ continuous and Lipschitz in $y$, unique solution exists
        \item[\textbf{Euler:}] $y_{n+1} = y_n + hf(t_n,y_n)$. Local error $O(h^2)$, global $O(h)$. Stable for $h|\lambda| \leq 2$
        \item[\textbf{Heun (RK2):}] $k_1 = f(t_n,y_n)$; $k_2 = f(t_n+h,y_n+hk_1)$; $y_{n+1} = y_n + \frac{h}{2}(k_1+k_2)$. Order 2
        \item[\textbf{RK4:}] $k_1 = f(t_n,y_n)$; $k_2 = f(t_n+\frac{h}{2},y_n+\frac{h}{2}k_1)$; $k_3 = f(t_n+\frac{h}{2},y_n+\frac{h}{2}k_2)$; $k_4 = f(t_n+h,y_n+hk_3)$; $y_{n+1} = y_n + \frac{h}{6}(k_1+2k_2+2k_3+k_4)$. Local $O(h^5)$, global $O(h^4)$
        \item[\textbf{Implicit Euler:}] $y_{n+1} = y_n + hf(t_{n+1},y_{n+1})$. A-stable, order 1
    \end{description}

    \subsection*{Linear Multistep}
    \begin{description}
        \item[\textbf{General:}] $\sum_{j=0}^k \alpha_j y_{n+j} = h\sum_{j=0}^k \beta_j f_{n+j}$ (Explicit if $\beta_k=0$, implicit otherwise).
        \item[\textbf{Adams-Bashforth:}] $y_{n+1} = y_n + h\sum_{j=0}^{k-1} \gamma_j f_{n-j}$
              \begin{itemize}
                  \item AB1: $y_{n+1} = y_n + hf_n$ (Euler)
                  \item AB2: $y_{n+1} = y_n + h[\frac{3}{2}f_n - \frac{1}{2}f_{n-1}]$
                  \item AB3: $y_{n+1} = y_n + h[\frac{23}{12}f_n - \frac{16}{12}f_{n-1} + \frac{5}{12}f_{n-2}]$
              \end{itemize}
        \item[\textbf{Adams-Moulton:}] $y_{n+1} = y_n + h\sum_{j=0}^k \gamma_j^* f_{n+1-j}$
              \begin{itemize}
                  \item AM1: $y_{n+1} = y_n + hf_{n+1}$ (Backward Euler)
                  \item AM2: $y_{n+1} = y_n + \frac{h}{2}(f_{n+1} + f_n)$ (Trapezoidal, A-stable)
                  \item AM3: $y_{n+1} = y_n + \frac{h}{12}(5f_{n+1} + 8f_n - f_{n-1})$
              \end{itemize}
        \item[\textbf{Method of Order $q$:}] Local truncation error $\tau_{n+1} = C_q h^{q+1} y^{(q+1)}(\xi) + O(h^{q+2})$ where
              \[C_q = \frac{1}{q!}\left(\sum_{j=0}^k \alpha_j j^q - q \sum_{j=0}^k \beta_j j^{q-1}\right)\]
              If $C_0 = C_1 = \cdots = C_{q-1} = 0$ and $C_q \neq 0$, method has order $q$.
        \item[\textbf{Zero stability:}] If roots of char. poly $\rho(z) = \sum_{j=0}^k \alpha_j z^j = 0$ satisfy $|z_j| \leq 1$ and roots on unit circle have multiplicity 1 (i.e., $\rho'(z_j) \neq 0$ for $|z_j|=1$), method is zero stable
        \item[\textbf{Convergence:}] Zero stability + consistency $\Rightarrow$ convergence (Dahlquist equivalence theorem)
        \item[\textbf{A-stability:}] Method stable for all $h\lambda$ with $\Re(\lambda) \leq 0$. AM2 (trapezoidal) is A-stable
    \end{description}

    % --- Column 4 ---
    \section*{Optimization (Brief)}
    \begin{description}
        \item[\textbf{Gradient:}] $\nabla f(x)$ gives steepest ascent direction
        \item[\textbf{Hessian:}] $H(x)$ contains second derivatives
        \item[\textbf{Steepest descent:}] $x_{k+1} = x_k - \alpha_k \nabla f(x_k)$ with line search for $\alpha_k$
        \item[\textbf{Newton:}] Solve $H(x_k)p_k = -\nabla f(x_k)$, set $x_{k+1} = x_k + p_k$. Requires positive definite $H$
    \end{description}

    \section*{Error Analysis \& Stability}
    \begin{description}
        \item[\textbf{Error sources:}] Modelling, discretization (truncation), rounding
        \item[\textbf{Floating point:}] $\text{fl}(a \circ \mathbf{b}) = (a \circ \mathbf{b})(1+\delta)$, $|\delta| \leq \varepsilon_{\text{mach}}$
        \item[\textbf{Absolute vs relative:}] Absolute: $|\tilde{x}-x|$, Relative: $\frac{|\tilde{x}-x|}{|x|}$
        \item[\textbf{Algorithm stability:}] Small perturbations $\Rightarrow$ small changes in result
        \item[\textbf{Conditioning vs stability:}] Conditioning = problem sensitivity to data; Stability = algorithm sensitivity
    \end{description}
\end{multicols*}

\end{document}
