\documentclass[../../main.tex]{subfiles}
\begin{document}
\section{Eksamen 2024}
\subsection*{Oppgave 1}
La $u \in H^2(\Omega)$ være en løsning av Poisson-likningen:
\begin{align*}
    -\Delta u                 & = f \quad \text{i } \Omega, \quad \text{hvor} f \in L^2(\Omega) \\
    \nabla u \cdot \mathbf{n} & = 0 \quad \text{på } \partial \Omega, \quad \text{(Neumann)}
\end{align*}

\subsubsection*{(a) Svak formulering}
La $u, v \in V = H^1(\Omega)$. Først så ganger vi med testfunksjonen $v$ og integrerer over $\int_\Omega$:
\[
    \int_{\Omega}-\Delta u v \, dx  = \int_{\Omega}f v \, dx
\]
Vi vet at:
\begin{align*}
    \nabla \cdot v \nabla u & =
    \begin{bmatrix}
        \partial_x \\ \partial_y \\ \partial_z
    \end{bmatrix}
    \cdot
    \left(
    v\begin{bmatrix}
             u_x & u_y & u_z
         \end{bmatrix}
    \right)                                                                                \\
                            & = v_x u_x + v_y u_y + v_z u_z + v (u_{xx} + u_{yy} + u_{zz}) \\
                            & = \nabla v \cdot \nabla u + v \Delta u                       \\
    -v \Delta u             & = \nabla v \cdot \nabla u - \nabla\cdot(v\nabla u)
\end{align*}
som gir:
\begin{align*}
    \int_{\Omega}-\Delta u v \, dx & = \int_{\Omega} \nabla v \cdot \nabla u \, dx - \int_{\Omega} \nabla\cdot(v\nabla u) \, dx
\end{align*}
Fra greens theorem om divergens:
\[
    \int_{\Omega} \nabla \cdot (v\nabla u) dx = \int_{\partial \Omega} v\nabla u \cdot \mathbf{n} \, dS
\]
Som i vårt tilfelle er gitt av Neumann betingelsen er $0$ s.a.
\begin{align*}
    \int_{\Omega}-\Delta u v \, dx & = \int_{\Omega} \nabla v \cdot \nabla u \, dx
\end{align*}
som til slutt gir oss den bilineære formen og linære funksjonen:
\[
    a(u, v) := \int_{\Omega} \nabla v \cdot \nabla u \, dx \qquad g(v) := \int_{\Omega}f v \, dx \qed
\]

\subsubsection*{(b) V-ellipsitet, og unik løsning av PDE}
V-ellipsitet er definert som:
\[
    \exists \alpha > 0 \quad \text{s.a.} \quad a(v,v) \geq \alpha \|v\|_V^2 \quad \forall \; v \in V
\]
Som egentlig bare betyr at den bilineære formen strengt nedre bundet, og er positiv definitt.

for å vise at $a(,)$ har en unik løsning må vi vise to ting: at den er \textbf{Koersiv (V-ell.)} og \textbf{kontinuerlig} i $\Omega$.

\begin{description}
    \item[Koersiv:] La $v \in V$:
          \begin{align*}
              |a(v,v)| & = \int_{\Omega} |\nabla v|^2 \, dx \\
                       & = \|\nabla v\|_0^2
          \end{align*}
          Fra definisjonen av normen til $V$ har vi:
          \[
              \|v\|_V^2 = \|v\|_0^2 + \|\nabla v\|_0^2
          \]
          og Poincaré:
          \[
              \|v\|_0^2 \leq C_\Omega \|\nabla v\|_0^2
          \]
          som gir:
          \begin{align*}
              \|v\|_V^2                                    & \leq (1 + C_\Omega)\|\nabla v\|_0^2 \\
                                                           & = (1 + C_\Omega) |a(v,v)|           \\
              \Rightarrow \frac{1}{1 + C_\Omega} \|v\|_V^2 & \leq |a(v,v)|
          \end{align*}
          Altså er $a(,)$ V-ellipsisk med $\alpha = \frac{1}{1 + C_\Omega}$.
    \item[Kontinuitet:] La $u, v \in V$:
          \begin{align*}
              |a(u,v)| & = \left|\int_{\Omega} \nabla u \cdot \nabla v \, dx\right|       \\
                       & \leq \int_{\Omega} |\nabla u||\nabla v| \, dx                    \\
                       & \leq \|\nabla u\|_0 \|\nabla v\|_0 \quad \text{(Cauchy-Schwarz)} \\
                       & \leq \|u\|_V \|v\|_V
          \end{align*}
          Altså er $a(,)$ kontinuerlig med konstant $C=1$.
\end{description}
\textbf{Lax-Milgram teorem} sier at hvis $a(,)$ er V-ellipsisk og kontinuerlig, og $g(v)$ er en lineær og kontinuerlig funksjon på $V$, så eksisterer det en unik løsning $u \in V$ slik at:
\[
    a(u,v) = g(v) \quad \forall v \in V
\]

\subsection*{Oppgave 2}
\subsubsection*{(a) Ciarlet-trippel}

Et finit element er en triple $(K,\mathcal P,\mathcal N)$ hvor:
\begin{enumerate}
    \item $K\subset\mathbb{R}^n$ er et begrenset domene med stykkevis glatt rand (elementet).
    \item $\mathcal P$ er et endelig-dimensjonalt rom av funksjoner på $K$ (formfunksjoner).
    \item $\mathcal N=\{N_1,\dots,N_k\}$ er en samling lineære funksjonaler på $\mathcal P$ som danner en basis for $\mathcal P^*$, slik at kartet
          $v\mapsto (N_i(v))_{i=1}^k$ gir en isomorfi $\mathcal P\cong\mathbb{R}^k$ (nodalverdiene bestemmer funksjonen).
\end{enumerate}
For å illustrere Ciarlet-trippelet, la oss se på et konkret eksempel med et lineært triangulært element:

\begin{figure}[htbp]
    \centering
    \begin{tikzpicture}[scale=2]
        % Triangle vertices
        \coordinate (A) at (0,0);
        \coordinate (B) at (2,0);
        \coordinate (C) at (1,1.5);

        % Draw triangle
        \draw[thick, fill=state-blue!50] (A) -- (B) -- (C) -- cycle;

        % Mark vertices (nodes)
        \fill[red] (A) circle (1pt);
        \fill[red] (B) circle (1pt);
        \fill[red] (C) circle (1pt);

        % Add coordinate labels
        \node[below, font=\small] at (A) {$v_1=(0,0)$};
        \node[below, font=\small] at (B) {$v_2=(1,0)$};
        \node[above, font=\small] at (C) {$v_3=(0.5,1)$};

        % Domain label
        \node at (1,0.75) {\textbf{Element} $K$};

        % Annotations with arrows
        \draw[->, thick] (2,1.5) -- (C);
        \node[right, align=left] at (2,1.5) {\textbf{Noder} $\mathcal{N}$:\\
            $N_i(v) = v(v_i)$\\
            (punktevaluering)};

        \draw[->, thick] (1,-0.5) -- (1,0);
        \draw[->, thick] (1, -0.5) -- (0.5,0.75);
        \draw[->, thick] (1, -0.5) -- (1.5,0.75);
        \node[below, align=center] at (1,-0.5) {\textbf{Formfunksjoner} $\mathcal{P}$:\\
            $\mathcal{P}_1 = \text{span}\{1, x, y\}$\\
            (lineære polynomer)};
    \end{tikzpicture}
    \caption{Ciarlet-trippel $(K,\mathcal{P},\mathcal{N})$ for lineært triangulært element}
\end{figure}

I dette eksemplet:
\begin{itemize}
    \item $K$: referansetriangel med hjørner $(0,0)$, $(1,0)$, $(0.5,1)$
    \item $\mathcal{P} = \mathcal{P}_1(K) = \{a + bx + cy : a,b,c \in \mathbb{R}\}$ (lineære polynomer)
    \item $\mathcal{N} = \{N_1, N_2, N_3\}$ hvor $N_i(v) = v(v_i)$ (punktevaluering i hjørnene)
\end{itemize}

De tre nodalverdiene $v(v_1)$, $v(v_2)$, $v(v_3)$ bestemmer entydig enhver lineær funksjon på triangelet.

\subsubsection*{(b) Kubisk Lagrange-element i 2D}
$(K, \mathcal{P}, \mathcal{N})$ der:
\begin{itemize}
    \item $K \subset \mathbb{R}^2$ er et trekantet domene med hjørner $v_1, v_2, v_3$.
    \item $\mathcal{P} = \mathcal{P}_3(K)$ er rommet av polynomer på $K$ med grad $\leq 3$.
    \item $\mathcal{N} = \{N_1, N_2, N_3, N_4, N_5, N_6, N_7, N_8, N_9, N_{10}\}$ hvor $N_i(v) = v(v_i)$ (punktevaluering i hjørnene)
\end{itemize}
Nodene er plassert som følger:
\begin{itemize}
    \item Hjørner: $v_1, v_2, v_3$
    \item Midtpunkter på sidene: $m_{12}, m_{23}, m_{31}$
    \item Interne punkter: $p_1, p_2, p_3, p_4$
\end{itemize}
Dette gir totalt 10 noder, som samsvarer med dimensjonen til $\mathcal{P}_3(K)$.
Nodalverdiene $v(v_i)$ for $i=1,\ldots,10$ bestemmer entydig enhver kubisk funksjon på trekanten $K$.


\begin{figure}[htbp]
    \centering
    \begin{tikzpicture}[scale=1.5]
        % Triangle vertices
        \coordinate (A) at (0,0);
        \coordinate (B) at (3,0);
        \coordinate (C) at (1.5,2.6);

        % Draw triangle
        \draw[thick, fill=state-blue!20] (A) -- (B) -- (C) -- cycle;

        % Corner nodes
        \fill[red] (A) circle (2pt);
        \fill[red] (B) circle (2pt);
        \fill[red] (C) circle (2pt);

        % Edge nodes (2 nodes per edge for cubic)
        \coordinate (AB1) at (1,0);
        \coordinate (AB2) at (2,0);
        \coordinate (BC1) at (2.25,1.3);
        \coordinate (BC2) at (1.875,1.95);
        \coordinate (CA1) at (1.125,1.95);
        \coordinate (CA2) at (0.75,1.3);

        \fill[blue] (AB1) circle (1.5pt);
        \fill[blue] (AB2) circle (1.5pt);
        \fill[blue] (BC1) circle (1.5pt);
        \fill[blue] (BC2) circle (1.5pt);
        \fill[blue] (CA1) circle (1.5pt);
        \fill[blue] (CA2) circle (1.5pt);

        % Internal node
        \coordinate (INT) at (1.5,1.3);
        \fill[green] (INT) circle (1.5pt);

        % Labels for corner nodes
        \node[below left, font=\small] at (A) {$v_1$};
        \node[below right, font=\small] at (B) {$v_2$};
        \node[above, font=\small] at (C) {$v_3$};

        % Labels for edge nodes
        \node[below, font=\tiny] at (AB1) {$m_1$};
        \node[below, font=\tiny] at (AB2) {$m_2$};
        \node[right, font=\tiny] at (BC1) {$m_3$};
        \node[above right, font=\tiny] at (BC2) {$m_4$};
        \node[above left, font=\tiny] at (CA1) {$m_5$};
        \node[left, font=\tiny] at (CA2) {$m_6$};

        % Label for internal node
        \node[below right, font=\tiny] at (INT) {$p_1$};

        % Domain label
        \node at (1.5,0.7) {\textbf{Element} $K$};

        % Legend
        \node[align=left, font=\small] at (4.5,2) {
            \textcolor{red}{\textbullet} Hjørnenoder (3)\\
            \textcolor{blue}{\textbullet} Kantnoder (6)\\
            \textcolor{green}{\textbullet} Intern node (1)\\
            Total: 10 noder
        };

        % Annotation
        \node[below, align=center, font=\small] at (1.5,-0.5) {
            $\mathcal{P}_3(K) = \text{span}\{1, x, y, x^2, xy, y^2, x^3, x^2y, xy^2, y^3\}$\\
            $\dim(\mathcal{P}_3(K)) = 10$
        };
    \end{tikzpicture}
    \caption{Kubisk Lagrange-element i 2D med 10 noder}
\end{figure}

\end{document}
