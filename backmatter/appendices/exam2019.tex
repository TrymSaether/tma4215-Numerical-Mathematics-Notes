\documentclass[a4paper,10pt]{article}
\usepackage{../../macros/trymtex}

\begin{document}

\section{Exam S2019}

\subsection{Problem 6: Interpolatory Quadrature}

\begin{enumerate}[label=\textbf{\arabic*.}, leftmargin=*, itemsep=1em]
    \item
    \begin{enumerate}
        \item \textbf{Setup}
        \[
        I[f] = \int_{-1}^{1} f(x)\, w(x)\, dx, \qquad
        Q_n[f] = \sum_{j=0}^{n} \alpha_j\, f(x_j),
        \]
        where
        \begin{itemize}
            \item \textbf{Nodes:} \(x_0, \dots, x_n \in [-1,1]\) are pairwise distinct,
            \item \textbf{Weight:} \(w : [-1,1] \to [0, \infty)\) is integrable.
        \end{itemize}

        \item \textbf{Lagrange Cardinal Polynomials}
        \[
        \ell_j(x) = \prod_{\substack{k=0 \\ k \neq j}}^{n} \frac{x - x_k}{x_j - x_k},
        \qquad
        \ell_j(x_k) = \delta_{jk}, \qquad j = 0, \dots, n.
        \]
        The family \(\{\ell_j\}_{j=0}^{n}\) forms a basis for the space \(\mathbb{P}_n\) of polynomials of degree at most \(n\).
        \begin{itemize}
            \item \emph{Tip:} Memorise the Kronecker property \(\ell_j(x_k) = \delta_{jk}\); it simplifies sums immediately.
        \end{itemize}

        \item \textbf{Determining the Weights (Integrate-the-Basis Trick)}
        \begin{itemize}
            \item Exactness on each \(\ell_i\) requires
            \[
            Q_n[\ell_i] = I[\ell_i], \qquad i = 0, \dots, n.
            \]
            Compute:
            \begin{align*}
            Q_n[\ell_i]
                &= \sum_{j=0}^{n} \alpha_j\, \ell_i(x_j)
                 = \sum_{j=0}^{n} \alpha_j\, \delta_{ij}
                 = \alpha_i,  \\
            I[\ell_i]
                &= \int_{-1}^{1} \ell_i(x)\, w(x)\, dx.
            \end{align*}
            Thus,
            \[
            \boxed{
            \alpha_i = \int_{-1}^{1} \ell_i(x)\, w(x)\, dx,
            \quad i = 0, \dots, n.
            }
            \]
            \item \emph{Tip:} No linear systems! Just integrate the cardinal functions.
        \end{itemize}

        \item \textbf{Verifying Exactness for All \(p \in \mathbb{P}_n\)}
        \begin{itemize}
            \item Let \(p \in \mathbb{P}_n\). Interpolate \(p\) in the \(\ell\)-basis:
            \[
            p(x) = \sum_{k=0}^{n} p(x_k)\, \ell_k(x).
            \]
            Integrate with the weight \(w\):
            \begin{align*}
            I[p]
                &= \int_{-1}^{1} p(x)\, w(x)\, dx
                 = \int_{-1}^{1} \sum_{k=0}^{n} p(x_k)\, \ell_k(x)\, w(x)\, dx \\
                &= \sum_{k=0}^{n} p(x_k) \int_{-1}^{1} \ell_k(x)\, w(x)\, dx
                 = \sum_{k=0}^{n} p(x_k)\, \alpha_k
                 = Q_n[p].
            \end{align*}
            Therefore, \(Q_n[p] = I[p]\) for every polynomial \(p\) of degree at most \(n\).
        \end{itemize}

        \item \textbf{Conclusion}
        \begin{itemize}
            \item With
            \[
            \alpha_i = \int_{-1}^{1} \ell_i(x)\, w(x)\, dx,
            \quad i = 0, \dots, n,
            \]
            the interpolatory \((n+1)\)-point rule \(Q_n\) achieves degree of exactness \(n\), \emph{regardless of how the nodes are chosen}.
            \item \emph{Final tip:} The same construction appears in Gaussian quadrature: if you choose the nodes as the zeros of the orthogonal polynomial of degree \(n+1\), the degree of exactness jumps to \(2n+1\). Remember this connection—it is an easy extra mark if asked ``why Gauss nodes are special''.
        \end{itemize}
    \end{enumerate}
\end{enumerate}

\end{document}
